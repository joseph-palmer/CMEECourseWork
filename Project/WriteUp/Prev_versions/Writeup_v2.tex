\documentclass[11pt,usenames,dvipsnames]{article}
\usepackage{lineno}
\usepackage{graphicx}
\usepackage[left=3cm, right=3cm, top=2cm]{geometry}
\usepackage{float}
\usepackage{caption}
\usepackage[round]{natbib}
\usepackage{pgfgantt}
\usepackage{amsmath}
\usepackage{booktabs}
\usepackage{amsfonts}
\usepackage[hidelinks]{hyperref} % for underscores in bibtex
\bibliographystyle{plainnat}
\captionsetup[table]{skip=12pt}
\linespread{1.3}
\setlength{\parskip}{2em}
\newcommand{\Lagr}{\mathcal{L}}
\newcommand{\lagr}{\mathcal{l}}
\DeclareMathOperator\erf{erf}
\begin{document}
\begin{titlepage}
\begin{center}
	\large{MSC. Computational Methods in Ecology and Evolution }\\
	\textbf{Thesis}\\[0cm]
	\huge{\line(1,0){380}\\
		Examining foraging distance distributions of the Western Honeybee, \textit{Apis mellifera}: A comparison between rural and urban environments\\
	\line(1,0){380}}\\[2cm]
\end{center}


\begin{minipage}[t]{0.5\textwidth}
\begin{flushleft}
	\Large{\textbf{Author}}\\
	\large{ Joseph Palmer\\
		CID: 01613406}\\
	joseph.palmer18@imperial.ac.uk\\[1cm]	
\end{flushleft}
\end{minipage}
\begin{minipage}[t]{0.5\textwidth}
\begin{flushright}
	\Large{\textbf{Supervisors}}\\
	\large{ Professor Vincent A.A. Jansen}\\
	Vincent.Jansen@rhul.ac.uk\\
	\large{Dr Elli Leadbeater}\\
	Elli.Leadbeater@rhul.ac.uk\\
	\large{Dr Richard J. Gill}\\
	r.gill@imperial.ac.uk
\end{flushright}
\end{minipage}

\vspace{1cm}
\begin{center}
	\large{\textbf{Imperial College London }}\\[0.2cm]
	\large{\textit{Department of Life Sciences, Imperial College London,}\\
		\textit{Ascot, Berkshire, SL5 7PY, United Kingdom}}\\[0.5cm]
	
	\large{\textbf{Royal Holloway University of London }}\\[0.2cm]
	\large{\textit{School of Biological sciences, Royal Holloway University of London,\\
			Egham Hill, Egham, Surrey, TW20 0EX, United Kingdom}}\\[0.5cm]
	\textbf{Word count:} 100 [replace]\\[0.5cm]
	\textbf{August 2019}
\end{center}

\end{titlepage}
\newpage
\tableofcontents
\newpage


\section{Introduction}

The western honey bee (\textit{Apis mellifera}) is among the most important pollinators for both natural and agricultural ecosystems \citep{Albrecht2018}. Living in highly advanced eusocial colonies with overlapping generations and division of labour, honey bee colonies function as a single reproductive unit. With a typical colony size of 30,000 individuals, in temperate environments a typical colony must collect around 120 Kg of nectar and 20 Kg of pollen \citep{Seeley1995}. Consequentially, the representative foraging area of a colony is estimated at around 100 Km\textsuperscript{2} with 95\% of foraging trips occurring within 6 Km of the hive \citep{Samuelson2017}. The acquisition of these resources from the environment is done by worker bees, which share the responsibility of optimising foraging effort so the colony can grow \citep{Samuelson2017}. In order to assist with resource acquisition, honey bees employ the waggle dance to convey information regarding the location of resources to nest mates. This offers us a unique window into the foraging behaviour of the colony as a whole and allows us to assess movement at different scales in this model central place pollinator (CPP). 

Under optimal foraging theory, research into CPP has primarily focused of the mechanisms governing foraging behaviour. Optimal foraging theory suggests that for all organisms individual foragers should maximise energy intake whilst minimising loss. Consequentially, natural selection should produce efficient foraging strategies. When applied to CPP the dynamics change as resources need to be returned to a central place, effectively anchoring the worker to only forage within a certain area. As the colony can be thought of as a superorganism \citep{Holldobler2009} the optimal foraging distance for a worker is dependent not only upon its own needs but also on that of the colony. 

The primary objective of a worker honey bee is to retrieve food to increase the growth rate of the colony. The more workers foraging for food the more biomass and with it gene copies the colony can accumulate, thereby increasing the fitness of the individual worker. In order for the colony to grow the worker must acquire more energy for the colony than it costs for it to retrieve the food source. Due to the waggle dance, information about the location of resources allows the colony to minimise search time, reducing the energetic load placed on workers. This creates an interesting dynamic. 

As workers have access to information on the location of resources, colony foraging should converge on profitable patches. However, whilst a strategy of allocating all workers to a particular patch is potentially very rewarding, over multiple foraging bouts it is more profitable to allocate only a fraction of workers to a particular patch as energetic losses compound over multiple bouts, reducing the colonies net energy gain to zero. This dynamic is well documented in game theory in the presence of information (cite). In the context of a honey bee colony, workers are not allocated by a central decision maker and function relatively autonomously. Therefore, foraging decisions reflecting colony needs should be considered at both the individual and colony level. This assertion is supported by \cite{Hendriksma2019} in their study on honey bees and bumble bees \textit(Bombus terrestris). The authors found altering colony protein and carbohydrate stores produced both colony level responses, measured by the number of foragers allocated to nectar and pollen foraging, and individual responses, as measured by pollen or nectar preference and loading. 

Given foraging dynamics appear dependent upon not only the demands on the colony but also its perception by individuals, to what extent variations in colony foraging distance reflect these complex dynamics is an important question. Specifically, the effect of these complexities on the scales at which the colony forage is yet to be considered. Scale, in terms of movement, is characterised by movements which differ by an order of magnitude \citep{Levin1992}. For example, whilst movement patterns are often classified according to purpose, such as migratory, foraging, mate search, and territory searching movements, what often underlies these differences is the scale at which they occur. 

Movements within a given movement type can also occur at different scales. For example, in the study of optimal searching strategies a myriad of publications have, often controversially \citep{Viswanathan1996}, proposed evidence of scale free movement in foraging behaviour \citep{Harris2012, Ariel2015, Humphries2010, Baronchelli2013, Boyer, Ramos-Fernandez2004, Sims2008, Viswanathan1999}. Scale free movement, often called a L\`evi process, is composed of multiple short flights interspersed with decreasingly probable longer flights (figure 1). The resulting distributions are characterised by a heavy tailed distribution and are classed as scale free as they have no characteristic scale \citep{Reynolds2018}. This theory was given mechanistic support by \cite{Shlesinger1986} in their recognition that L\`evi flights can provide an advantage for searching due to the reduced revisitation rates for previously explored areas. Consequentially, such processes are widely considered to represent an optimal searching strategy \citep{Viswanathan1999,Humphries2014} as opposed to Brownian processes which occur at a single scale (figure 1). 

\begin{figure}[H]
	\centering
	\includegraphics[scale=0.7]{LeviFlight.jpg}
	\caption{\textbf{a}) Brownian motion showing equal contribution of step length to the average, \textbf{b}) L\`evi flight ($\gamma = 2$) with increased frequency of longer steps. Taken from \cite{Barthelemy2008}.}
\end{figure}

Despite assertions that scale free movement could arise under optimal foraging theory, the exact conditions favouring such a strategy, as well as whether or not L\`evi processes represent an emergent or evolved characteristic, are debated \citep{Wosniack2017, Pyke2015, Kolzsch2015, DeJager2013}. In addition, despite optimal foraging theory producing different dynamics in CPP, the role of scale in their foraging movements has yet to be evaluated at the colony level. Consequently, evaluating colony foraging movement in terms of scale could offer insights into the complex dynamics of collective foraging. In order to do this, however, different approaches are needed to evaluate scale than those used for individual movement.

The general approach to test for scale free movements has been proposed by \cite{Murphy2007} yet the idea of L\`evi processes remains controversial \citep{Pyke2015}. This is further compounded by evidence that scale free movement can arise from a concatenation of bouts occurring at different scales. For example, in questioning the findings of a study on muscle movements by \cite{DeJager2011}, \cite{Jansen2012} identified muscle movement is best described by a composite Brownian walk, instead of a L\`evi flight. This finding of increased evidence of composite Brownian motion has also been seen in \cite{Petrovskii2011, Sakamoto2017, Gautestad2012, Zhao2016}. The composite Brownian motion can be thought of as a mid point between processes operating at a single scale and those with no identifiable scale. 

Following on from studies identifying composite Brownian motion, scale can be characterised by the rate parameter of the exponential distribution \citep{Petrovskii2011}. Consequently, Brownian motion is characterised by movement along a single exponential, scale free movement possesses a theoretically infinite number of rates and composite Brownian motion is described by a limited number of rates greater than 1. The mechanistic rational behind composite Brownian motion is that they divide movement into separate groups, such as an intensive phase associated with moving around a small area and a relaxed phase characterised by movement between patches \citep{Auger-Methe2015}. Whilst these methods provide an intermediate between Brownian motion and scale free movement, the number of exponentials used to identify the number of scales acting in the data are relatively low, often less than 4 \citep{Sakamoto2017, Zhao2016}. 

Given the suggestion by \cite{Petrovskii2011} that individuals move at different scales due to physiological differences, it is possible that limiting the number of exponentials in a composite random walk underestimates the number of scales potentially acting in a population. This could also be the case for honey bees. Variations in individual workers perception of how much they need to forage could theoretically lead to movements at different scales. In addition, although individual differences between honey bees should be low given their high genetic similarity, variations in access to food during development may cause variations in morphology (cite). Consequently, analysing colony level foraging distance in terms of scale allows a phenomenological exploration of colony level foraging behaviour. 

In order to fully analyse colony foraging movement in terms of scale, we propose a new method which extends the work of \cite{Petrovskii2011} to a sum of $n$ exponentials (SumExp). In doing so, the relative contributions of movements at different scales can be identified by weighting each exponential in the sum. This permits the identification of the amount of movement at a given scale without limiting the number of scales under consideration. This approach, theoretically at least, produces a more fine grained view of movements in terms of scale. In accordance with \cite{Murphy2007}, our approach uses likelihood based methods to fit the SumExp. However, unlike \cite{Petrovskii2011} and others, the strength of our method comes in its ability to visualise data in terms of scale rather than arbitrarily fitting discrete scaling models.

In order to evaluate the effects of scale on the distribution of colony level foraging behaviour, whilst we can take approaches from the field of movement biology, the interpretations need to be considered differently. In movement ecology the presence of different scales indicates different movement strategies, e.g. Brownian motion (\textit{single exponential, gamma and Gaussian}), composite Brownian motion (\textit{sum of greater than 1 exponentials}) and scale free motion (\textit{sum of n exponentials}), the same cannot be said for colony level foraging distances. Although composed of foraging bouts of individual workers, at the colony level because workers return to a central place the resulting distribution does not describe the colonies movement through the environment, rather it describes patterns of resource acquisition \citep{Visscher1982, Waddington1994, Couvillon2014, Couvillon2015}. As such, the probability density function (PDF) of the distribution of foraging distances for a colony is similar in nature to a seed dispersal kernel. In order to identify how plant seeds disperse, distance is used to form a dispersal kernel which can then be analysed statistically to identify underlying patterns between species \citep{Bullock2017}. We argue a similar approach can be used with foraging distances in order to evaluate colony movement statistically. 

To investigate the role of scale in colony level foraging dynamics, foraging distances were decoded from waggle dance observations in order to create a foraging distance kernel. This was conducted for colonies of honey bees at two sites, the first an agricultural-rural (agri-rural) setting and the second an urban green space (urban). These sites were chosen as they represent anthropogenically altered land types that are becoming increasingly abundant habitats for CPP. By statistically analysing these foraging distance kernels this study extends previous work to evaluate colony level honey bee foraging in terms of scale. In addition, because this method is centred around the use of an exponential function it is prudent to also compare the fits against other distributions. For this reason we also fit four distributions with different tail properties: the half-normal, log-normal, exponential and gamma distributions. By combining a novel method to identify colony foraging at different scales with common statistical distributions, this study aims to provide a phenomenological profile of honey bee foraging in the face of increasing anthropogenic pressures.

\section{Methods}

\subsection{Data collection}
Foraging distances were collected by decoding waggle dance observations from honey bees in two sites, one in an agri-rural setting the other in an urban green space. The hives were set up as two three-frame observation hives of standard size in order to record waggle dances. Recordings took place for between two and four hours twice a month from April to September 2017. Waggle dance observations were collected and converted to longitude and latitude coordinates by Ash Samuelson as part of their PhD research at Royal Holloway University of London, under the supervision of Dr Ellie Leadbeater and Dr Richard Gill.

\subsection{Calculation of distance from waggle dance observations}
Foraging distances were calculated as the euclidean distance between the hive and decoded waggle dance coordinates. This was done through equations ?, ? and ?, where a and b are the hive and destination latitudes respectively, c and d are the hive and destination longitudes and k is the conversion constant for kilometres, 6371.

% foraging distance equations from coordinates
\begin{equation}
x = f_{(abcd)} = \sin\left(\frac{b - a}{2}\right)^2\ +\ \cos(a)\ \cos(b) \sin\left(\frac{d - c}{2}\right)^2 
\end{equation}
\begin{equation}
y = 2\ \text{atan2}(\sqrt{x}, \sqrt{1 - x})
\end{equation}
\begin{equation}
D_e = K\ y
\end{equation}

\subsection{Distribution fitting}

In order to identify the most probable distributions explaining foraging distance, 4 distributions were chosen as candidates: Exponential, Gamma, Half-normal and Lognormal (equations ?, ?, ? and ? respectively). These distributions were chosen as they represent a sample of light, exponential and half-normal, and heavy, gamma and lognormal, tailed common distributions.

\begin{equation}
%exponential pdf
\lambda e^{-\lambda x}
\end{equation}
\begin{equation}
%exponential ccdf
1 - e^{-\lambda x}
\end{equation}
\begin{equation}
%Gamma pdf
\frac{1}{\Gamma(k)\theta^k}x^{k-1}e^{-\frac{x}{\theta}}
\end{equation}
\begin{equation}
%Gamma ccdf
1 - \frac{1}{\Gamma(k)}\gamma(k,\frac{x}{\theta})
\end{equation}
\begin{equation}
%halfnormal pdf
\frac{\sqrt{2}}{\sigma \sqrt{\pi}} \exp \left(-\frac{x^2}{2 \sigma^2}\right)
\end{equation}
\begin{equation}
%halfnormal ccdf
\erf\left(\frac{x}{\sigma \sqrt{2}}\right)
\end{equation}
\begin{equation}
%lognormal pdf
\frac{1}{x \sigma \sqrt{2 \pi}}\ e^{- \frac{(\ln\ x - \mu)^2}{2\sigma^2}}
\end{equation}
\begin{equation}
%lognormal ccdf
\frac{1}{2} + \frac{1}{2}\ \erf\left[\frac{\ln\ x - \mu}{\sqrt{2 \sigma}}\right]
\end{equation}

To determine the parameters which best fit the distributions to the data we used maximum likelihood. This was conducted in Python using the Scipy package \textit{minimize} and the Sequential Least Squares Programming (SLSQP) method. The function minimises an objective function through gradient decent. Consequentially, we provided the method with the negative log-likelihood equations for each distribution in order to identify the parameters producing the maximum log-likelihood estimate (see supplementary equations ?, ?, ? and ?). Once the parameters with the highest likelihood are identified we fit the model to the complimentary cumulative distribution function (CCDF). This was chosen over the probability density function (PDF) as the probability of obtaining an exact value is 0, where as the CCDF allows us to view the probability of obtaining a value greater than or equal to a particular value (see supplementary equations ?, ?, ? and ?). 


\subsection{Sum of exponentials procedure}

The sum of exponentials (SumExp) procedure is composed of $n$ exponential functions (equation ?) summed together, each constrained by a weighting factor (equation ?, subject to equations ? and ?). The parameters of the model are $\lambda$ and $\psi$, where $\lambda$ is the rate component of the exponential and $\psi$ is the weighting factor which influences the relative contribution of an exponential with a given rate in the sum. $\lambda$ values are restricted to only positive real numbers (equation ?) and both individual $\psi$ and the combined sum of $\psi$ values must be greater than or equal to 0 and less than or equal to 1 (equation ?).

\begin{equation}
%single exponential equation
\lambda e^{-\lambda x}
\end{equation}
\begin{equation}
% sum of exponentials equation
f(x) = \sum_{i=1}^{n-1} \psi_i \lambda_i e^{-\lambda_i x} + \left(1 - \sum_{i=1}^{n-1}\psi_i\right) \lambda_n e^{-\lambda_n x}
\end{equation}
subject to (eq ?, ? \& ?)
\begin{equation}
% constraints on individual psi
0\leq \psi_i \leq 1
\end{equation}
\begin{equation}
% constraints on sum of psi
0\leq \sum_{i=1}^{n-1}\psi_i \leq 1
\end{equation}
\begin{equation}
% constraints on individual lambda
\lambda = \mathbb{R}^+
\end{equation}

\subsection{Numerical optimisation}

In order to identify the most probable parameter values for the SumExp model we used numerical optimisation to estimate the parameters with maximum likelihood. The procedure for doing this is to take the product of Sumexp over the number of observations (equation ?) and find the peak in the resulting likelihood curve. For easier computation the log-likelihood is used herein (equation ?). The analysis was conducted in Python using the scipy.optimize package \textit{minimize} (ref) and the Sequential Least Squares Programming (SLSQP) routine (ref) to identify the minima of the negative log-likelihood function. The SLSQP method uses gradient decent to scale the likelihood curve until the model derivative is within a given tolerance of 0. As such, the gradient function of partial derivatives (equation ?) is provided to improve convergence.

\begin{equation}
%likelihood equation for SumExp
\Lagr_{(\psi|x)} = \prod_{i=1}^{n} f(x)
\end{equation} 
\begin{equation}
%loglikelihodd equation for SumExp
\ell_{(\psi|x)} = \sum_{i=1}^{n} \ln f(x)
\end{equation} 
\begin{equation}
% gradient function of partial derivitives of SumExp. 
\nabla h(x) = \begin{bmatrix} -\sum_{i=1}^{n} [\frac{\lambda_0 e^{-\lambda_0 x_i}}{f(x_i)} - \frac{\lambda_n e^{-\lambda_n x_i}}{f(x_i)}] \\
... \\
... \\
-\sum_{i=1}^{n} [\frac{\lambda_{n-1} e^{-\lambda_{n-1} x_i}}{f(x_i)} - \frac{\lambda_n e^{-\lambda_n x_i}}{f(x_i)}] \\
\end{bmatrix}
\end{equation}

The optimisation procedure used involves selecting multiple rates evenly spaced over a given interval to create a SumExp of a given size. These rates are then fixed in the model but the weights are left as free parameters within their bounds and constraints. If the rates are free in the model along with the weights then the procedure would identify the most probable sum of exponentials explaining the data. Instead, by keeping the rates fixed we allow the weights to act as switches that limit the contribution of an associated rate. By varying the upper and lower bounds as well as the number of rates (resolution) the procedure can be used to explore the data in terms of scale and aid the development of more parsimonious models. 

\subsection{Experimental error}
The distributions of data in our study showed distances bellow 1Km, which corresponds to approximately less than 1 second of waggle dance duration. Whilst it is entirely plausible such short dances are representative of foraging locations, it is important to consider possible measurement errors influencing the data. To consider this we present results of both the full data set and the data excluding points bellow 1Km. 

\section{Results}

\noindent
\subsection{Simulated data}

\subsubsection{Single exponential}
In order to test if the numerical method can identify the sum of exponentials with the most likely parameter estimates, we tested data sampled from a single exponential with $\lambda = 1.8$. Sampling 1,000 data points we calculated the analytical rate parameter, derived as the reciprocal of the data mean, as 1.82 with an associated maximum likelihood estimate (MLE) of -401.28. The sum of exponentials method identified a distribution with a single dominant cluster of 2 peaks at rates 1.83 and 1.85 and associated weights of 0.93 and 0.059; explaining 99\% of the observations. The remaining 1\% is explained by a peak at $\lambda = 0.86$ (figure ?). The associated MLE is -401.26. Therefore, although the likelihood is slightly improved, the added parameters indicate the single exponential is the more parsimonious model.

\begin{figure}[H]
	\centering
	\includegraphics[scale=1]{../Results/Plots/SyntheticData_1exp.pdf}
	\caption{Scale spectrum of linearly selected rates for data sampled from a single exponential with $\lambda = 1.8$.}
\end{figure}


\subsubsection{Multiple exponentials}

\noindent
\textbf{Equal weighting (50/50)}

In order to test if the method can identify data derived from processes operating at two different scales, we sampled data points from a sum of two exponentials with rates 1.8 and 4.4 and equal weights. With a sample size of 1,000 the method identified five rates in 3 district clusters with $\lambda$ approximately 1.26, 2.68 and 6.49 and associated $\psi$ of approximately 0.11, 0.78 and 0.11 (Table ?, figure ?). With 10,000 observations the rates form two main clusters around rates 1.87 and 4.6 with associated $\psi$ values of 0.55 and 0.42. At 100,000 observations the rate peaks appear close to the actual input values with $\psi$ matching the input weightings (table ?). 

\begin{table}[H]
	\centering
	\caption{Numerically optimised rates ($\lambda$) and weights ($\psi$) with data sampled from $n$ observations of a sum of two exponentials with $\lambda = 1.8,\ 4.4$ and $\psi = 0.5$.}
	\begin{tabular}{rrr}
\toprule
    $n$ &  $\lambda$ &    $\psi$ \\
\midrule
   1000 &       1.24 &  0.003290 \\
   1000 &       1.30 &  0.111582 \\
   1000 &       2.68 &  0.777452 \\
   1000 &       6.46 &  0.002152 \\
   1000 &       6.52 &  0.105525 \\
  10000 &       1.00 &  0.003992 \\
  10000 &       1.84 &  0.144445 \\
  10000 &       1.90 &  0.416288 \\
  10000 &       4.60 &  0.415916 \\
  10000 &       9.94 &  0.019359 \\
 100000 &       1.78 &  0.280559 \\
 100000 &       1.84 &  0.218817 \\
 100000 &       4.36 &  0.304631 \\
 100000 &       4.42 &  0.195990 \\
\bottomrule
\end{tabular}

\end{table}


\begin{figure}[H]
	\centering
	\includegraphics[scale=1]{../Results/Plots/SyntheticData_2exp1844.pdf}
	\caption{Scale spectrum of linearly selected rates for data sampled from a sum of two exponentials with $\lambda = 1.8,\ 4.4$ and $\psi = 0.5$. Upper bound $= 10$, lower bound $= 0$, resolution $= 150$ \textbf{A}) $n = 1,000$, \textbf{B}) $n = 10,000$,  \textbf{C}) $n = 100,000$.}
\end{figure}

\noindent
\textbf{Unequal weighting (70/30)}

Using the same parameters as above but with unequal weightings of 0.3 and 0.7 for rates 1.8 and 4.4 respectively, we again tested how the method performs with different data sizes. With a data size of 1,000 the method identified two main peaks around 1.6 and 3.85 with corresponding $\psi$ values of 0.18 and 0.73. For 10,000 observations the main peaks are at 2.02 and 4.09 with associated $\psi$ of 0.33 and 0.6 with the remainder at rates 1.0 and 9.94. With 100,000 observations the peaks lie at 1.71 and 4.39, just as with equal weighting (figure ?, table?) but the weighting is 0.3 and 0.7 respectively (figure?, table?).

\begin{table}[H]
	\centering
	\caption{Numerically optimised rates ($\lambda$) and weights ($\psi$) with data sampled from $n$ observations of a sum of two exponentials with $\lambda = 1.8,\ 4.4$ and $\psi = 0.5$.}
	\begin{tabular}{rrr}
\toprule
    $n$ &  $\lambda$ &    $\psi$ \\
\midrule
   1000 &       1.54 &  0.047849 \\
   1000 &       1.60 &  0.129053 \\
   1000 &       3.82 &  0.352723 \\
   1000 &       3.88 &  0.382955 \\
   1000 &       3.94 &  0.087421 \\
  10000 &       1.00 &  0.008807 \\
  10000 &       2.02 &  0.331551 \\
  10000 &       4.06 &  0.238486 \\
  10000 &       4.12 &  0.360444 \\
  10000 &       9.94 &  0.060711 \\
 100000 &       1.78 &  0.248727 \\
 100000 &       1.84 &  0.048693 \\
 100000 &       4.36 &  0.543943 \\
 100000 &       4.42 &  0.155287 \\
 100000 &       9.94 &  0.003350 \\
\bottomrule
\end{tabular}

\end{table}

\begin{figure}[H]
	\centering
	\includegraphics[scale=1]{../Results/Plots/SyntheticData_2exp1844_73.pdf}
	\caption{Scale spectrum of linearly selected rates for data sampled from a sum of two exponentials with $\lambda = 1.8,\ 4.4$ and $\psi = 0.5$. Upper bound $= 10$, lower bound $= 0$, resolution $= 150$ \textbf{A}) $n = 1,000$, \textbf{B}) $n = 10,000$,  \textbf{C}) $n = 100,000$.}
\end{figure}

\noindent
\textbf{Five rate sum of exponentials}

In order to test how the method responds to processes operating at multiple scales we sampled data from a five rate sum of exponential. As the previous results indicate the method can over fit to data, we used very different rates to represent processes with multiple scales scanning several orders of magnitude. The rates used are 1.8, 4.4, 7.5, 12.5 and 16.7 with equal weightings of 0.2. With a sample size of 1,000 the method identified 4 peaks with rates of 1.5, 2.74, 9.65 and 22.5 and corresponding $\psi$ of 0.05, 0.29, 0.38 and 0.28 (table ?, figure ?). 10,000 observations returned 3 main peaks at rates 2.15, 4.89 and 12.6 with corresponding $\psi$ values of 0.24, 0.23, and 0.48, with the remainder located at two other peaks of 1.0 and 29.9. With 100,000 observations the method identified four main peaks at rates 1.79, 4.3, 7.34 and 15.43 with corresponding $\psi$ values of 0.21, 0.14, 0.29 and 0.35 (table ? figure ?). 

\begin{table}[H]
	\centering
	\caption{Numerically optimised rates ($\lambda$) and weights ($\psi$) with data sampled from $n$ observations of a sum of five exponentials with $\lambda = 1.8,\ 4.4,\ 7.5,\ 12.5,\ 16.7$ and $\psi = 0.2$.}
	\begin{tabular}{rrr}
\toprule
    $n$ &  $\lambda$ &    $\psi$ \\
\midrule
   1000 &   1.497143 &  0.050242 \\
   1000 &   2.740000 &  0.285434 \\
   1000 &   9.617143 &  0.128375 \\
   1000 &   9.700000 &  0.195719 \\
   1000 &   9.782857 &  0.181994 \\
   1000 &   9.865714 &  0.094545 \\
   1000 &  22.377143 &  0.003371 \\
   1000 &  22.460000 &  0.020353 \\
   1000 &  22.542857 &  0.019953 \\
   1000 &  22.625714 &  0.015760 \\
   1000 &  22.708571 &  0.004230 \\
  10000 &   1.000000 &  0.009865 \\
  10000 &   2.077143 &  0.045273 \\
  10000 &   2.160000 &  0.194265 \\
  10000 &   4.894286 &  0.234190 \\
  10000 &  12.517143 &  0.080454 \\
  10000 &  12.600000 &  0.216137 \\
  10000 &  12.682857 &  0.188127 \\
  10000 &  29.917143 &  0.031690 \\
 100000 &   1.745714 &  0.006408 \\
 100000 &   1.828571 &  0.203963 \\
 100000 &   4.314286 &  0.126473 \\
 100000 &   4.397143 &  0.017868 \\
 100000 &   7.214286 &  0.001503 \\
 100000 &   7.297143 &  0.146539 \\
 100000 &   7.380000 &  0.143362 \\
 100000 &  15.334286 &  0.050094 \\
 100000 &  15.417143 &  0.136866 \\
 100000 &  15.500000 &  0.132326 \\
 100000 &  15.582857 &  0.034596 \\
\bottomrule
\end{tabular}

\end{table}
\begin{figure}[H]
	\centering
	\includegraphics[scale=1]{../Results/Plots/SyntheticData_2exp1844_5.pdf}
	\caption{Scale spectrum of linearly selected rates for data sampled from a sum of five exponentials with $\lambda = 1.8,\ 4.4,\ 7.5,\ 12.5,\ 16.7$ and $\psi = 0.2$. Upper bound $= 30$, lower bound $= 1$, resolution $= 350$. \textbf{A}) $n = 1,000$, \textbf{B}) $n = 10,000$,  \textbf{C}) $n = 100,000$.}
\end{figure}

What is notable here is that the method was unable to identify the input rates and appropriate weights at any of the data sizes tested. Although they converged towards to the true values as sample size was increased, the procedure omitted the rate 12.5. This indicates the method struggles to identify process occurring at more than 3 or 4 scales without very large data sets, possibly due to overlaps in parameter space (figure ?).

\begin{figure}[H]
	\centering
	\includegraphics[scale=1]{../Results/Plots/SyntheticData_Hist_5.pdf}
	\caption{Overlapping histograms of data sampled from exponentials with different rates. \textbf{A}) $\lambda = 1.8, 4.4$, \textbf{B}) $\lambda = 1.8, 4.4, 7.5$, \textbf{C}) $\lambda = 1.8, 4.4, 7.5, 12.5$, \textbf{D}) $\lambda = 1.8, 4.4, 7.5, 12.5, 16.7$.}
\end{figure}


\subsection{Analysis of foraging distance}

The agri-rural data consists of 193 observations ranging from 0.016 to 5.17Km with a mean of 1.29. The urban data consists of 221 observations ranging from 0.0006 to 3.18Km with a mean of 0.85. Combined these datasets consist of 414 observations ranging from 0.0006 to 5.17Km with a mean of 1.05.

\begin{figure}[H]
	\centering
	\includegraphics[scale=1]{../Results/Plots/DistributionHist.pdf}
	\caption{Histogram of agri-rural (\textbf{A}), urban (\textbf{B}) and combined (\textbf{C}) foraging distances.}
\end{figure}

With values bellow 1Km removed and the remaining data normalised by subtracting 1 from the distances, the agri-rural data consists of 107 observations ranging from 0.025 to 4.17Km with a mean of 0.85. The urban data consists of 81 observations ranging from 0.006 to 2.18Km with a mean of 0.53. Combined these datasets consist of 188 observations ranging from 0.0006 to 4.17Km with a mean of 0.71.

\begin{figure}[H]
	\centering
	\includegraphics[scale=1]{../Results/Plots/DistributionHist_1km.pdf}
	\caption{Histogram of agri-rural (\textbf{A}), urban (\textbf{B}) and combined (\textbf{C}) foraging distances above 1Km, normalised by subtracting 1Km from results.}
\end{figure}

In order to accommodate for the difference in distributions between these two data sets, foraging distances were compared using a bootstrapped hypothesis test. This was chosen as a non-parametric equivalent to the t-test as the data violates the condition of normality (figure ? histogram), whilst retaining a high degree of statistical power (ref on bootstrapping). For both the truncated and full data sets, foraging distance differed significantly between environments (Non-parametric bootstrap test: simulations = 10,000. full data: agri-rural mean = 1.29, urban mean 0.85, degrees of freedom = 192 and 220, t = 5.70, p $\le$ 0.001; truncated data: agri-rural mean = 0.85, urban mean 0.53, degrees of freedom = 106 and 80, t = 3.34, p < 0.01) (figure ?). 

\begin{figure}[H]
	\centering
	\includegraphics[scale=0.9]{../Results/Plots/DistDiff.pdf}
	\caption{Comparison of agri-rural and urban mean foraging distances. With both datasets foraging distance is significantly different between environments (Non-parametric bootstrap test: simulations = 10,000) \textbf{A}) Full data, agri-rural mean = 1.29, urban mean 0.85, degrees of freedom = 192 and 220, t = 5.70, p < 0.001. \textbf{B}) Truncated data, agri-rural mean = 0.85, urban mean 0.53, degrees of freedom = 106 and 80, t = 3.34, p $\le$ 0.01}
\end{figure}

\subsection{Fitting distributions using maximum likelihood}

\subsubsection{Full data}
In order to determine the distribution which best describes the data, maximum likelihood methods were used to fit a number of candidate distributions. Each distribution was fit to the urban and agri-rural data and the fit assessed using the Akaike information criterion (AIC) and associated Akaike weights (AICw). For the rural data the most parsimonious model is the gamma distribution (table ?) with an AIC score seven points lower than the next best fitting model, the half-normal distribution. Using Akaike weights we can determine the probability that a given model is the best model of our candidate set. By dividing the AICw of the best model by that of the next best candidate (0.978/0.0233), the Gamma distribution is approximately 41.9 times more likely to be the best model in terms of the Kullback–Leibler discrepancy than the half-normal model.

For the urban data, the best fitting model is the half-normal distribution with an AIC score 14 points lower than the next best fitting model, the gamma distribution (table ?, figure ?). The interpretation of Akaike weights indicates the half-normal is approximately 1225.5 times more likely to represent the best model in terms of the Kullback–Leibler discrepancy than the Gamma distribution.

For the agri-rural and urban foraging data combined, the best fitting model is the half-normal with an AIC score 10 points lower than the next best fitting model, the gamma distribution (table ?, figure ?). The interpretation of Akaike weights indicates the half-normal is approximately 196.1 times more likely to represent the best model in terms of the Kullback–Leibler discrepancy than the Gamma distribution.

\begin{table}[H]
	\centering
	\caption{AIC and weighted AIC scores for distributions fit using maximum likelihood to Agri-rural foraging distances.}
	\begin{tabular}{llrr}
\toprule
Distribution &      MLE &         AIC &          AICw \\
\midrule
       Gamma & -222.258 &  448.516283 &  9.766669e-01 \\
 Half-normal & -226.992 &  455.984833 &  2.333305e-02 \\
 Exponential & -242.106 &  486.211336 &  6.373377e-09 \\
   Lognormal & -244.337 &  492.674635 &  2.516994e-10 \\
\bottomrule
\end{tabular}

\end{table}
\begin{table}[H]
	\centering
	\caption{AIC and weighted AIC scores for distributions fit using maximum likelihood to urban foraging distances.}
	\begin{tabular}{llrr}
\toprule
Distribution &      MLE &         AIC &          AICw \\
\midrule
 Half-normal & -174.376 &  350.751452 &  9.991404e-01 \\
       Gamma & -180.487 &  364.973596 &  8.153183e-04 \\
 Exponential & -184.401 &  370.801177 &  4.424703e-05 \\
   Lognormal & -215.753 &  435.506022 &  3.939176e-19 \\
\bottomrule
\end{tabular}

\end{table}
\begin{table}[H]
	\centering
	\caption{AIC and weighted AIC scores for distributions fit using maximum likelihood to combined argi-rural and urban foraging distances.}
	\begin{tabular}{llrr}
\toprule
Distribution &      MLE &         AIC &          AICw \\
\midrule
 Half-normal & -416.694 &  835.388612 &  9.949263e-01 \\
       Gamma & -420.973 &  845.945810 &  5.073695e-03 \\
 Exponential & -435.614 &  873.228880 &  6.037832e-09 \\
   Lognormal &  -483.13 &  970.260184 &  5.138081e-30 \\
\bottomrule
\end{tabular}

\end{table}

\begin{figure}[H]
	\centering
	\includegraphics[scale=1]{../Results/Plots/RuralDistributionFit.pdf}
	\caption{Complementary Cumulative Distribution Frequency (CCDF) of agri-rural foraging distances with fitted model lines. A) Exponential, B) Gamma, C) Half-normal, D) Lognormal.}
\end{figure}

\begin{figure}[H]
	\centering
	\includegraphics[scale=1]{../Results/Plots/UrbanDistributionFit.pdf}
	\caption{Complementary Cumulative Distribution Frequency (CCDF) of urban foraging distances with fitted model lines. A) Exponential, B) Gamma, C) Half-normal, D) Lognormal.}
\end{figure}

\begin{figure}[H]
	\centering
	\includegraphics[scale=1]{../Results/Plots/AllDistributionFit.pdf}
	\caption{Complementary Cumulative Distribution Frequency (CCDF) of combined agri-rural and urban foraging distances with fitted model lines. A) Exponential, B) Gamma, C) Half-normal, D) Lognormal.}
\end{figure}

\subsubsection{Truncated data}

\hspace{\parindent}
With distances lower than one Kilometre removed and the remaining distances normalised by subtracting 1, the agri-rural data shows the best fitting model is the gamma. However the difference in AIC is less than 1 point to the next best fitting model, the exponential (table ?). A comparison of the weighted AIC scores suggests the gamma is approximately 1.4 times more likely to be most parsimonious model than the exponential (table ?).

In contrast, the urban data shows the best model as the exponential, almost 2 AIC points lower than the next best fitting model, the gamma distribution (table ?). The comparison of AIC weights indicates the exponential is approximately 2.5 times more likely to be the best model than the gamma (table ?).

For both the agri-rural and urban foraging distances combined, the best model was identified as the exponential, 2 AIC points lower than the next best fitting model, the gamma distribution (table ?). The comparison of AIC weights indicates the exponential is approximately 2.5 times more likely to be the best model than the gamma (table ?). 

\begin{table}[H]
	\centering
	\caption{AIC and weighted AIC scores for distributions fit using maximum likelihood to Agri-rural foraging data greater than 1Km.}
	\begin{tabular}{llrr}
\toprule
Distribution &      MLE &         AIC &      AICw \\
\midrule
       Gamma & -88.4953 &  180.990608 &  0.444697 \\
 Exponential & -89.8349 &  181.669702 &  0.316665 \\
   Lognormal & -89.1227 &  182.245326 &  0.237468 \\
 Half-normal & -95.4361 &  192.872228 &  0.001170 \\
\bottomrule
\end{tabular}

\end{table}
\begin{table}[H]
	\centering
	\caption{AIC and weighted AIC scores for distributions fit using maximum likelihood to urban foraging data greater than 1Km.}
	\begin{tabular}{llrr}
\toprule
Distribution &      MLE &        AIC &      AICw \\
\midrule
 Exponential &  -29.446 &  60.891934 &  0.699804 \\
       Gamma & -29.3618 &  62.723513 &  0.280062 \\
 Half-normal & -32.9953 &  67.990502 &  0.020116 \\
   Lognormal &  -39.024 &  82.048053 &  0.000018 \\
\bottomrule
\end{tabular}

\end{table}
\begin{table}[H]
	\centering
	\caption{AIC and weighted AIC scores for distributions fit using maximum likelihood to combined argi-rural and urban foraging distances greater than 1Km.}
	\begin{tabular}{llrr}
\toprule
Distribution &      MLE &         AIC &          AICw \\
\midrule
 Exponential & -124.347 &  250.693054 &  7.134821e-01 \\
       Gamma & -124.259 &  252.517781 &  2.865158e-01 \\
   Lognormal & -136.396 &  276.792210 &  1.534700e-06 \\
 Half-normal & -138.436 &  278.871001 &  5.427747e-07 \\
\bottomrule
\end{tabular}

\end{table}

\begin{figure}[H]
	\centering
	\includegraphics[scale=1]{../Results/Plots/RuralDistributionFit_1km.pdf}
	\caption{Complementary Cumulative Distribution Frequency (CCDF) of agri-rural foraging distances over 1Km with fitted model lines. A) Exponential, B) Gamma, C) Half-normal, D) Lognormal.}
\end{figure}
\begin{figure}[H]
	\centering
	\includegraphics[scale=1]{../Results/Plots/UrbanDistributionFit_1km.pdf}
	\caption{Complementary Cumulative Distribution Frequency (CCDF) of urban foraging distances over 1Km with fitted model lines. A) Exponential, B) Gamma, C) Half-normal, D) Lognormal.}
\end{figure}
\begin{figure}[H]
	\centering
	\includegraphics[scale=1]{../Results/Plots/AllDistributionFit_1km.pdf}
	\caption{Complementary Cumulative Distribution Frequency (CCDF) of combined agri-rural and urban foraging distances over 1Km with fitted model lines. A) Exponential, B) Gamma, C) Half-normal, D) Lognormal.}
\end{figure}

\subsection{Sum of exponential fitting}

\subsubsection{Full data}

For the agri-rural data, using variations of fixed rates between 0 and 100, the method identified a single exponential with $\lambda = 0.77$ most likely contributed to the data. This matches the analytically derived single exponential rate value (figure ?, table ?). For the urban data a single peak is also identified around 1.18 again matching the analytical MLE for a single exponential (figure ?, table ?). When the data from each site is combined, the method identified a single peak at $\lambda = 0.94$ explained the data best, within 0.01 of the analytical MLE of the combined data (table ?, figure ?). 

\begin{table}[H]
	\centering
	\caption{Estimated rate ($\lambda$) and weight ($\psi$) sum of exponential parameters for agri-rural and urban foraging distances. Analytical $\lambda$ derived from MLE of single exponential.}
	\begin{tabular}{lrrr}
\toprule
   Location &  $\lambda$ &    $\psi$ &  Analytical $\lambda$ \\
\midrule
 Agri-rural &   0.771429 &  0.916771 &              0.775356 \\
 Agri-rural &   0.777143 &  0.083229 &              0.775356 \\
      Urban &   1.171429 &  0.084733 &              1.180111 \\
      Urban &   1.177143 &  0.915253 &              1.180111 \\
   Combined &   0.948571 &  1.000000 &              0.949131 \\
\bottomrule
\end{tabular}

\end{table}


\begin{figure}[H]
	\centering
	\includegraphics[scale=1]{../Results/Plots/Sumexp_rural_full.pdf}
	\caption{\textbf{A}) Scale spectrum of agri-rural foraging distances. Single peak identified with $\lambda = 0.77$, upper bound $= 2$, lower bound $= 0$, resolution $= 350$. \textbf{B}) Complementary Cumulative Distribution Frequency (CCDF) of agri-rural foraging distances. Black line is data, blue line is sum of exponentials fit.}
\end{figure}

\begin{figure}[H]
	\centering
	\includegraphics[scale=1]{../Results/Plots/Sumexp_urban_full.pdf}
	\caption{\textbf{A}) Scale spectrum of urban foraging distances. Single peak identified with $\lambda = 1.18$, upper bound $= 2$, lower bound $= 0$, resolution $= 350$. \textbf{B}) Complementary Cumulative Distribution Frequency (CCDF) of urban foraging distances. Black line is data, blue line is sum of exponentials fit.}
\end{figure}

\begin{figure}[H]
	\centering
	\includegraphics[scale=1]{../Results/Plots/Sumexp_comb_full.pdf}
	\caption{\textbf{A}) Scale spectrum of combined agri-rural and urban foraging distances. Single peak identified with $\lambda = 0.94$, upper bound $= 4$, lower bound $= 0$, resolution $= 350$. \textbf{B}) Complementary Cumulative Distribution Frequency (CCDF) of urban foraging distances. Black line is data, blue line is sum of exponentials fit.}
\end{figure}


\subsubsection{Truncated data}

For the agri-rural foraging data over 1Km, using variations of fixed rates between 0 and 100, the method identified a single peak around $\lambda = 1.17$ most likely contributed to the data. This matches the analytically derived single exponential rate value (figure ?, table ?). For the urban foraging data over 1Km the method identified two peaks around rates 1.86 and 46.9. The peak around 1.86 is the dominant peak, explaining 97\% of the data with the remainder produced by the secondary peak. Consequentially, the second peak is small in figure ? (figure ?, table ?). The dominant peak is 0.03 larger than the analytical MLE of a single exponential for this data, 1.189. The associated likelihoods between the sum of exponentials and the single exponential is -29.081 and -29.446 respectively, however, this is not significantly different and indicates the single exponential is the more parsimonious model (Sum of exponentials $MLE = -29.08\ AIC = 76.2, AIC_w < 0.001$, single exponential $MLE = -29.46\ AIC = 60.9, AIC_w > 0.999$, table ?). 

Combined, the foraging data from both sites showed two peaks around rates 1.36 and 2.08, explaining 90\% and 10\% of the data respectively (table ?, figure ?). The likelihood for the sum of exponential and a single exponential is 0.02 higher than for the single exponential, however this is not significant indicating the single exponential is the more parsimonious model (SumExp $MLE = -124.33,\ AIC = 284.67,\ AIC_w < 0.001$, single exponential $MLE = -124.35,\ AIC = 250.69,\ AIC_w > 0.999$, table ?).

\begin{table}[H]
	\centering
	\caption{Estimated rate ($\lambda$) and weight ($\psi$) sum of exponential parameters for agri-rural, urban and combined foraging distances. Analytical $\lambda$ derived from MLE of single exponential.}
	\begin{tabular}{lrrr}
\toprule
   Location &  $\lambda$ &    $\psi$ &  Analytical $\lambda$ \\
\midrule
 Agri-rural &   1.154286 &  0.038405 &              1.174006 \\
 Agri-rural &   1.160000 &  0.158546 &              1.174006 \\
 Agri-rural &   1.165714 &  0.225814 &              1.174006 \\
 Agri-rural &   1.171429 &  0.241893 &              1.174006 \\
 Agri-rural &   1.177143 &  0.208409 &              1.174006 \\
 Agri-rural &   1.182857 &  0.126933 &              1.174006 \\
      Urban &   1.818182 &  0.913226 &              1.889797 \\
      Urban &   1.909091 &  0.055472 &              1.889797 \\
      Urban &  46.636364 &  0.002272 &              1.889797 \\
      Urban &  46.727273 &  0.003751 &              1.889797 \\
      Urban &  46.818182 &  0.004365 &              1.889797 \\
      Urban &  46.909091 &  0.004139 &              1.889797 \\
      Urban &  47.000000 &  0.004755 &              1.889797 \\
      Urban &  47.090909 &  0.007131 &              1.889797 \\
      Urban &  47.181818 &  0.004889 &              1.889797 \\
   Combined &   1.285714 &  0.008211 &              1.402957 \\
   Combined &   1.300000 &  0.067885 &              1.402957 \\
   Combined &   1.314286 &  0.106875 &              1.402957 \\
   Combined &   1.328571 &  0.128086 &              1.402957 \\
   Combined &   1.342857 &  0.134152 &              1.402957 \\
   Combined &   1.357143 &  0.127452 &              1.402957 \\
   Combined &   1.371429 &  0.110134 &              1.402957 \\
   Combined &   1.385714 &  0.087323 &              1.402957 \\
   Combined &   1.400000 &  0.063454 &              1.402957 \\
   Combined &   1.414286 &  0.047699 &              1.402957 \\
   Combined &   1.428571 &  0.024425 &              1.402957 \\
   Combined &   2.042857 &  0.006489 &              1.402957 \\
   Combined &   2.057143 &  0.014659 &              1.402957 \\
   Combined &   2.071429 &  0.019576 &              1.402957 \\
   Combined &   2.085714 &  0.020986 &              1.402957 \\
   Combined &   2.100000 &  0.018636 &              1.402957 \\
   Combined &   2.114286 &  0.012281 &              1.402957 \\
   Combined &   2.128571 &  0.001677 &              1.402957 \\
\bottomrule
\end{tabular}

\end{table}

\begin{table}[H]
	\centering
	\caption{Model statistics for urban foraging distances greater than 1Km.}
	\begin{tabular}{lrrr}
\toprule
  Model &        MLE &        AIC &      AICw \\
\midrule
    Exp & -29.445967 &  60.891934 &  0.999517 \\
 SumExp & -29.080624 &  76.161249 &  0.000483 \\
\bottomrule
\end{tabular}

\end{table}

\begin{table}[H]
	\centering
	\caption{Model statistics for combined agri-rural and urban foraging distances greater than 1Km.}
	\begin{tabular}{lrrr}
\toprule
  Model &         MLE &         AIC &          AICw \\
\midrule
    Exp & -124.346527 &  250.693054 &  1.000000e+00 \\
 SumExp & -124.334338 &  284.668675 &  4.190710e-08 \\
\bottomrule
\end{tabular}

\end{table}

\begin{figure}[H]
	\centering
	\includegraphics[scale=1]{../Results/Plots/Sumexp_rural_trun.pdf}
	\caption{\textbf{A}) Scale spectrum of agri-rural foraging distances. Single peak identified with $\lambda = 1.17$, upper bound $= 2$, lower bound $= 0$, resolution $= 350$. \textbf{B}) Complementary Cumulative Distribution Frequency (CCDF) of agri-rural foraging distances. Black line is data, blue line is sum of exponentials fit.}
\end{figure}

\begin{figure}[H]
	\centering
	\includegraphics[scale=1]{../Results/Plots/Sumexp_urban_trun.pdf}
	\caption{\textbf{A}) Scale spectrum of urban foraging distances. Two peaks identified around $\lambda = 1.86$ and $46.9$ with associated $\psi = 0.97$ and $0.03$, upper bound $= 50$, lower bound $= 0$, resolution $= 550$. \textbf{B}) Complementary Cumulative Distribution Frequency (CCDF) of urban foraging distances. Black line is data, blue line is sum of exponentials fit.}
\end{figure}

\begin{figure}[H]
	\centering
	\includegraphics[scale=1]{../Results/Plots/Sumexp_comb_trun.pdf}
	\caption{\textbf{A}) Scale spectrum of combined foraging distances from agri-rural and urban sites. Two peaks identified with $\lambda = 1.36$ and $2.08$ with associated $\psi = 0.9$ and $0.1$, upper bound $=5$, lower bound $= 0$, resolution $= 350$. \textbf{B}) Complementary Cumulative Distribution Frequency (CCDF) of combined foraging distances from agri-rural and urban sites. Black line is data, blue line is sum of exponentials fit.}
\end{figure}

\section{Discussion}

\subsection{Synthetic Data}
In order to build confidence in the results of this new method when applied to biological data, it is important to demonstrate the method can retrieve known rates from synthetic data. By sampling data from a single exponential distribution we are able to test the desired switch behaviour of the weights. That the method identified a single rate explained almost all of the data, closely resembling the analytically derived rate indicates the weighting parameters function as desired. Although approximately 1\% of the data was accounted for by a separate rate, that the SumExp likelihood was higher than the single exponential likelihood indicates the weights indeed lowered the contribution of incorrect rates. The fact they did not entirely rule out other peaks indicates the method is somewhat over fitting.

Whilst it is important our method can identify a single exponential, the primary use of this method is to describe processes that occur at multiple scales. Consequentially, it is prudent to ensure the method can identify instances with multiple rates. Fitting the method to data sampled from a sum of exponentials allows the results to be compared to known values. With 1,000 replicates the method identified a different number of peaks at different rates to those used to generate the data. That the method converged towards the input parameters as sample size was increased suggests there is a significant influence of sample size in convergence.

In addition to deriving the relative contributions of different rates, it is important to see the response to unequal weightings. This is akin to processes operating at multiple scales with different relative contributions. With a 70/30 split the method found the input weighting and rates with a large enough sample size. As lower sample sizes returned different results, this again highlights the influence of sample size in the ability of the method to identify the input parameters.

In testing the ability of the method to identify processes occurring a multiple scales, samples from five rates did not show the same convergence with sample size as the previous tests did. Whilst increasing sample size showed the method converged more towards the input values, the effect of increasing the number of rates resulted in a much larger sample size being required. Even with a large sample size the method still returned fewer rates than was actually used for sampling, suggesting the method has the potential to under estimate the number of rates underlying the data. 

Overall, our testing methods demonstrate the method responds well to data generated from processes occurring at different scales. However, our tests show sample size and rate has a significant effect on the ability of the method to identify the actual underlying scale values. These observations indicate random variations scale with both sample size and the number of underlying scales in the data. The effect of sample size is similar for our process as it is for parametric statistical tests based upon the normal distribution of errors. However, increasing the number of sampling rates exacerbates the problem. This is visually described well by the overlapping histograms in figure ?. Here, despite a sample size of 100,000, due to the overlap with exponentials of different scales the probability that a given data point originates from a given scale is reduced. Consequentially, the set of possible candidate sum of exponentials increases substantially with every new rate added. That the method identified an alternative combination of rates and weights therefore indicates the confidence region around the 'true' rates is large. Future research should continue to develop the statistical inference of this method to improve the convergence on the 'true' underlying scale values. Nevertheless, whilst these results show the difficulty in identifying underlying rates in data, that our method identified multiple rates demonstrates its suitability for exploring data in terms of scale.

\subsection{Honey bee foraging analysis}

In applying the SumExp method to honey bee foraging data we observed urban and agri-rural hives both forage at single scales which differed significantly between environments. However, caution should be taken before inferring ecological relevance from this observation as the method showed inconsistencies when used on this data. When the two data sets were combined, the resulting distribution is composed of observations from processes operating at different scales. Despite this, the method failed to identify this underlying difference in rates. As this was not observed in the truncated data it is unlikely this is a sample size issue. Instead, it is likely due to the data not being described well by an exponential distribution. As our method is based around the use of exponentials, it is likely that when supplied with data from different distributions the method fails to improve likelihood by adding in more exponentials.

Our observations of foraging distance is prone to error when identifying short distances. This is because the distance is derived from waggle dance duration at a conversion rate of approximately 1Km per second of dance. Consequentially, distances under 1Km indicate a short dance duration which are vulnerable to misinterpretation. To adapt for this we re-ran the analysis using distances only greater than 1Km. The resulting distributions appeared more exponential in nature and so are more applicable to the SumExp procedure. Using this truncated data we again observed honey bees from urban and agri-rural locations forage at different scales. The method suggested foraging occurs at a single scale at both locations, although there is limited evidence to indicate some movement at a second scale in the urban environment. On analysing the location data combined the method was able to identify two underlying scales. Although these differed from the analytical MLE of the two data sets, our tests with synthetic data suggest this is due to low sample sizes. Despite this, that our method was able to identify accurately the presence of multiple scales in the combined data reassures our findings that honey bees in both environments forage at a single scale. 

Both our results on the full honey bee foraging data and the data with short distances removed suggest honey bee foraging is not well approximated by a sum of exponentials, suggesting instead that honey bee foraging occurs at a single scale. Biologically, this indicates the colony as a whole foragers at the same scale. There are multiple possible explanations for this pattern. Although variations in honey bee perceptions of resource needs may well vary between individuals, as the resources required by the hive is mainly pollen and nectar the choice of what to forage for is limited. Consequently, over the many thousands of individuals in a colony, individual resource perception may balance out to accurately reflect colony needs. Nevertheless, this perception of colony need is theoretically continuous in these two dimensions of pollen and nectar, potentially causing variations in foraging effort. I.e. certain individual workers may be willing or able to invest more effort to forage further for a resource than others. This may result from differences between individuals. With overlapping generations, worker age is mixed. Some studies (cite) have shown that older workers make better foragers. Assuming this means they invest less energy in foraging they may be more likely to travel further for forage than younger bees as they forage more efficiently. Our results of foraging occurring at a single scale, however, indicating that whilst these differences may be present in the colony, they are not significantly different enough to cause a difference in scale. One other possible explanation for this is the land type our results were gathered on.

In an agri-rural setting, floral availability is limited by the presence of crops, many of which offer few resources to honey bees. Those that do offer resources are often only seasonally available. With crop and pasture land taking up most of the area, resources for pollinators are often restricted to hedge rows and small verge patches. Consequently, suitable foraging areas are constrained to specific locations, limiting choice. This limit in foraging area may homogenise the optimal foraging locations between individuals, causing the colony as a whole to forage at a single scale. Similarly, urban environments are typically characterised by human made structures devoid of floral resources for pollinators. Just as in the agri-rural setting, the localisation of floral patches to gardens, parks and verges may also explain the dominance of a single foraging scale by constraining workers to forager in a given area. 

As resource abundance is known to change over the season, scales of foraging may differ through time. Whilst our data was collected over the season, due to the shortage of data points at each month it was not feasible to explore scale within months using our method. That our results show a single scale operating in the data at each location indicates scale did not change throughout the season. However, the sample size of foraging events on a monthly time frame may have inhibited the methods ability to identify them. To investigate this further, more data over time is required for a given location.

As our data is not well approximated by a sum of exponentials, the fits of other distributions may highlight other mechanisms taking place. For the full data with no distances removed, agri-rural foraging is best described by a gamma distribution. The sum of $n$ exponential random variables with a mean rate of $\theta$ is a gamma random variable with shape $n$ and scale $\theta$ (equation ?). Consequentially, the gamma distribution can arise from processes occurring at multiple scales. However, that the SumExp method indicates a single scale acting in the data suggests the gamma forms from a sum of exponentials with the same rate value. In biological terms, this re-enforces the assertion that colony foraging behaviour occurs at a single scale.

\begin{equation}
% gamma as a sum of exponentials.
\sum_{i=1}^{n} \lambda e^{-\lambda x} \equiv \frac{1}{\Gamma(n)\lambda^n}x^{n-1}e^{-\frac{x}{\lambda}}
\end{equation}

With no data removed, the urban data is best described by a half-normal distribution. The half-normal distribution represents the absolute value of a normal distribution with mean 0 and variance limited to only positive values of the domain $\theta \in [0,\infty] $. In the context of our study, this represents a light tailed distribution characterising foraging distances which surrounds some mean value, as opposed to the exponential which describes a constantly decreasing probability. Consequentially, the parsimony of this distribution in the urban location indicates a single optimal foraging distance for urban honey bee foragers. This is possibly the result of greater physical barriers in the form of buildings than are present in the relatively open spaces of agri-rural settings. Isolated to a green oasis, the urban colony are potentially limited to forage within a certain range. However, green urban spaces such as parks, allotments and gardens can provide high floral resources throughout the season \citep{Baldock2015, Baldock2019, Plascencia2017}. This indicates forage availability is potentially high enough in these environments to allow for a short foraging distance. Both of these mechanisms could produce an optimal mean foraging distance with variations around it, as captured by the half-normal distribution. This could be tested in a future study by providing a colony of a given size with enough resources to grow at a given distance. Then, by varying the number of obstacles the effects could be compared. 

In conclusion, whilst our results can be interoperated mechanistically, sample size acted as a key limitation for our method. Further research is needed to improve the method in order to handle lower sample sizes. However, given this increases substantially with the number of rates under consideration, future studies of scale on colony level foraging behaviour should not be conservative with data quantity. Despite these limitations, our results demonstrate the SumExp routine is effective at analysing distance in terms of scale and indicates variations in landscape type influence the scale at which honeybees forage. Further theoretical research is needed to adequately consider the effects of scale on optimal foraging in CPP. In addition, future empirical studies focusing on the influence of three-dimensional landscape barriers will heighten our understanding of the behaviour of CPP. Combined, improving methods to evaluate scale in CPP has the potential to highlight otherwise hidden foraging behavioural mechanisms and allow for more effective management of urban and agri-rural environments for CPP. 


\bibliographystyle{plainnat}
\bibliography{Project_bib}
\end{document}





