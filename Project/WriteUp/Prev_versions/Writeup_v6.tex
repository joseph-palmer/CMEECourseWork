\documentclass[11pt,usenames,dvipsnames,a4paper]{article}
\usepackage{lineno}
\usepackage{graphicx}
\usepackage[left=2cm, right=2cm, top=2cm, bottom=2cm]{geometry}
\usepackage{float}
\usepackage{caption}
\usepackage[round]{natbib}
% set fancy headers
\usepackage{fancyhdr}
\pagestyle{fancy}
\fancyhead[L]{\nouppercase{\rightmark}} %  sets the right head element 
\fancyhead[R]{}
\renewcommand{\headrulewidth}{0.5pt}
\renewcommand{\footrulewidth}{0.05pt}
% other packages
\usepackage{pgfgantt}
\usepackage{amsmath}
\usepackage{booktabs}
\usepackage{amsfonts}
\usepackage[hidelinks]{hyperref} % for underscores in bibtex
\bibliographystyle{plainnat}
\captionsetup[table]{skip=12pt}
\renewcommand{\baselinestretch}{1.5}
\setlength{\parskip}{2em}
\newcommand{\Lagr}{\mathcal{L}}
\newcommand{\lagr}{\mathcal{l}}
\DeclareMathOperator\erf{erf}
\begin{document}

\begin{titlepage}

\begin{figure}[H]
	\includegraphics[scale=0.13]{../Poster/Images/ICL_Logo.png}
\end{figure}
\vfill
\begin{center}
% report title
\noindent
\huge{
\textbf{Colony level foraging movement in the western honey bee: the importance of scale}
}\\[1.5cm]
\noindent\rule{\textwidth}{0.05pt}\\[0.2cm]
\noindent
\Large{
	% Author
	Joseph Paul George Palmer\\
	% Date
	29\textsuperscript{th} August 2019
}\\

\noindent\rule{\textwidth}{0.05pt}
\vfill\vfill\vfill
% Declaration and word count
\large{A thesis submitted for the partial fulfilment of the requirements for the degree of Master of Science at Imperial College London.\\[0.5cm]
	Formatted in the journal style \textit{Methods in Ecology and Evolution}.\\
	Submitted for the MSc. in Computational Methods in Ecology and Evolution.\\[1.5cm]
	\textbf{Word count:} 6000
}\\
\vfill\vfill
\end{center}

\end{titlepage}

\newpage

\begin{center}
	\Large{\textbf{Declaration}}
\end{center}
\noindent
\Large{\textbf{Supervisor}}

\noindent
The primary supervisor is Prof. Vincent Jansen of Royal Holloway University of London. The secondary, internal, supervisor is Dr. Richard Gill of Imperial College London.

\noindent
\Large{\textbf{Data}}

\noindent
Waggle dance observations were collected and converted to longitude and latitude coordinates by Ash Samuelson as part of their PhD research at Royal Holloway University of London, under the supervision of Dr. Elli Leadbeater and Dr. Richard Gill.

\noindent
\Large{\textbf{Model}}

\noindent
All model development was performed in conjunction with Prof. Vincent Jansen. The model definition was pioneered by Prof. Jansen whilst the numerical optimisation was designed and implemented by me.

\noindent
\Large{\textbf{Code}}

\noindent
All code for this project, including that used to create figures and optimise the model, was my own work.

\newpage
\tableofcontents
\newpage

\section{Summary - remove for write up and convert to abstract}

\begin{itemize}
	\item The aim of this study is to evaluate if there is a fundamental difference in the way honey bees explore different landscapes.
	\item To identify if honey bees behave differently in urban and agri-rural environments we looked at the scale of individual movement.
	\item Movement at different scales indicates different underlying processes.
	\item However, currently there are no methods to disentangle scales within a given data set. This study outlines a new method using a sum of n exponential distributions in order to identify the scales operating in the data and their relative contributions to the overall movement pattern.
	\item In applying this method to synthetic data we demonstrate how the method works to identify known scales.
	\item We fit the method to honey bee foraging data from two hives  in different locations, showing a single dominant scale underlying both environments.
	\item Although the two distributions operate at different scales, the presence of a single scale indicates that despite environmental differences the foraging strategies used are the same across these two environments.
	
\end{itemize}

\section{Abstract}

\begin{linenumbers}
Put an abstract in me!
\end{linenumbers}

\section{Introduction}

\begin{linenumbers}
\hspace{\parindent}
From searching for food to migrating vast distances, movement forms a central component of almost all animal life on earth. Despite substantial research into developing models of animal movement, for many crucial groups such as pollinators still little is known about their movements through the environment (cite cant find). Further still, it is not known to what extent foraging behaviours between environments differ in their underlying processes. Answering such questions requires statistical methods to disentangle processes in movement data \citep{Nathan2008}. Whilst substantial research into animal movement has advanced towards this objective, the ability to identify different processes underlying movement remains a challenge \citep{Patterson2017}. 

Whilst movement is a central component of many animal behaviours, the information contained is largely one-dimensional. In essence, movement is described by distances which are divided into bouts in a given direction. The distributions of these bout lengths can then be statistically evaluated using common distributions to determine the underlying process \citep{Murphy2007, Reynolds2018}. Central to this is the use of scale. Scale, in terms of movement, is characterised by movements which differ by an order of magnitude \citep{Levin1992}. For example, whilst movement patterns are often classified according to purpose, such as migratory, foraging or mate searching movements, what often underlies these differences is the scale at which they occur. Consequently, movement at different scales indicates different underlying processes \citep{Nathan2008}. 

Typically, singular scale movement describes a random behaviour, also known as Brownian motion \citep{Barthelemy2008} (figure 1a). Given obvious problems with inferring biological meaning from random events, the focus of many animal movement studies \citep{Harris2012, Ariel2015, Humphries2010, Baronchelli2013, Boyer, Ayala-Orozco2004, Sims2008, Viswanathan1999} has often been to explore movements with no identifiable scale, termed scale free or a L\'evy process \citep{Viswanathan1999}. L\'evy processes are denoted by multiple short bouts interspersed with decreasingly probable longer bouts \citep{Barthelemy2008} (figure 1b), giving a very heavy tailed distribution which is scale invariant \citep{Reynolds2018}.

\begin{figure}[H]
	\centering
	\includegraphics[scale=0.7]{LeviFlight.jpg}
	\caption{\textbf{a}) Brownian motion showing equal contribution of step length to the average, \textbf{b}) L\`evi flight ($\gamma = 2$) with increased frequency of longer steps. Taken from \cite{Barthelemy2008}.}
\end{figure}

In contrast to movements along a single scale, L\'evy processes have been used to portray an optimal search strategy due to the reduced revisitation rates for previously explored areas \citep{Viswanathan1999, Humphries2014}. More recent attention, however, has questioned both the methodology and rationales behind assertions of scale free movements \citep{DeJager2013, Jansen2012, Kolzsch2015, Wosniack2017}, leading to the development of composite Brownian motion to bridge the gap between singular and scale free movements. Mechanistically, this divides movement into separate groups, such as an intensive phase associated with moving around a small area and a relaxed phase characterised by movement between patches \citep{Auger-Methe2015}. However, variations between individuals have also been shown to create the impression of multi-scale movement. In investigating  black bean aphid (\textit{Aphis fabae Scopoli}) movements, \cite{Petrovskii2011} identified physiological variations between individuals showed multi-scale movement at the population level, demonstrating the power of using scale to capture underlying processes. Whilst such studies undoubtedly advance the field of animal movement, we still lack the general tools needed to identify underlying scales \citep{Nathan2008, Patterson2017}.

Building from composite Brownian motion, this study extends methods outlined in \cite{Petrovskii2011} and \cite{Jansen2012} in order to evaluate movement in terms of scale. Using a weighted sum of $n$ exponential distributions, we propose a method to identify the relative contributions of movements at different scales and thus identify the number of underlying processes. Although the amount of mechanistic interpenetration that can be gained from a distribution of movements alone is limited \citep{Bearup2016}, identifying the number of processes and their relative contributions to movements at a given scale provides a phenomenological description of movement. This can then be used to inform research into possible underpinning mechanisms. 

Whilst the methods presented herein could be applied to previously analysed data, central place pollinators (CPP) represent one of many animal groups which, to our knowledge, have been neglected in studies of animal movement. Under increasing anthropogenic pressures causing global pollinator declines \citep{Powney2019}, understanding how CPP explore different environments is key to developing more effective environmental practices. Differences in environment type are known to vary colony foraging areas \citep{Lecocq2015, Samuelson2018}, however, little is known about the processes that underlay foraging behaviour at the colony level. Whilst social and genetic studies have been shown to influence other behaviours \citep{Fewell1993, Page1998, Sherman2002, Jones2004, Gruter2009}, their relative effects on colony foraging distance is unknown. By evaluating scale in CPP foraging movements, this study takes an important step in facilitating such research by evaluating the number of processes operating in foraging movement. 

The lack of knowledge around pollinator movements is not surprising as evaluations require movement data from individuals, which is challenging to collect due to their small size and large ranges. Conveniently, one pollinator, the western honey bee (\textit{Apis mellifera}), assists with providing such information. Representing one of the most important pollinators for both natural and agricultural ecosystems \citep{Albrecht2018}, in temperate environments workers must collect around 120 Kg of nectar and 20 Kg of pollen to feed and grow a colony of around 30,000 individuals \citep{Seeley1995}. Consequently, the representative foraging area of a colony is estimated at around 100 Km\textsuperscript{2} with 95\% of foraging trips occurring within 6 Km of the hive \citep{Samuelson2017}. To optimise this massive foraging effort, workers conduct a 'waggle dance' to provide distance and direction information of a resource to nest mates. By eves dropping on these dances, we are able to build a profile of colony foraging in the local environment and evaluate scales to infer the number of processes underlying colony foraging.

To investigate scale in colony level foraging dynamics, foraging distances were decoded from waggle dance observations for colonies of honey bees at two sites: an agricultural-rural (agri-rural) setting and an urban green space (urban). In an expanding anthropogenic world such landscapes are becoming increasingly abundant habitats for honey bees (CITE?). By teasing out the underlying scales operating in these locations, our method allows us to evaluate if there exists a fundamental difference in how honey bees forage in different environments. In addition, four distributions with different tail properties are also fit to foraging distances to provide additional information on the possible mechanisms underlying observed scales and to compare with our new method.

Besides evaluating the scales of collective foraging behaviour, this study evaluates the performance of our procedure on simulated data so as to build confidence in its ability to disentangle movements occurring at different scales. Consequently, by combining a novel method to identify colony foraging at different scales with common statistical distributions, our results provide a phenomenological profile of honey bee foraging which can be used to guide further mechanistic studies.
\end{linenumbers}
	
\section{Methods}

\subsection{Data collection}

\begin{linenumbers}
\hspace{\parindent}
Foraging distances were collected by decoding waggle dance observations from honey bees in two sites, one in an agri-rural setting the other in an urban green space. The hives were set up as two three-frame observation hives of standard size in order to record waggle dances. Recordings took place for between two and four hours twice a month from April to September 2017. Waggle dance observations were collected and converted to longitude and latitude coordinates by Ash Samuelson as part of their PhD research at Royal Holloway University of London, under the supervision of Dr. Ellie Leadbeater and Dr. Richard Gill.
\end{linenumbers}

\subsection{Calculation of distance from waggle dance observations}

\begin{linenumbers}
\hspace{\parindent}
Foraging distances were calculated as the euclidean distance between the hive and decoded waggle dance coordinates. This was done through equations ?, ? and ?, where a and b are the hive and destination latitudes respectively, c and d are the hive and destination longitudes and k is the conversion constant for kilometres, 6371.
\end{linenumbers}
% foraging distance equations from coordinates
\begin{equation}
x = f_{(abcd)} = \sin\left(\frac{b - a}{2}\right)^2\ +\ \cos(a)\ \cos(b) \sin\left(\frac{d - c}{2}\right)^2 
\end{equation}
\begin{equation}
y = 2\ \text{atan2}(\sqrt{x}, \sqrt{1 - x})
\end{equation}
\begin{equation}
D_e = K\ y
\end{equation}

\subsection{Distribution fitting}

\begin{linenumbers}
\hspace{\parindent}
In order to identify the most probable distributions explaining foraging distance, 4 distributions were chosen as candidates: Exponential, Gamma, Half-normal and Lognormal (equations ?, ?, ? and ? respectively). These distributions were chosen as they represent a sample of light (exponential and half-normal) and heavy (gamma and lognormal) tailed common distributions.
\end{linenumbers}
\begin{equation}
%exponential pdf
\lambda e^{-\lambda x}
\end{equation}
\begin{equation}
%exponential ccdf
1 - e^{-\lambda x}
\end{equation}
\begin{equation}
%Gamma pdf
\frac{1}{\Gamma(k)\theta^k}x^{k-1}e^{-\frac{x}{\theta}}
\end{equation}
\begin{equation}
%Gamma ccdf
1 - \frac{1}{\Gamma(k)}\gamma(k,\frac{x}{\theta})
\end{equation}
\begin{equation}
%halfnormal pdf
\frac{\sqrt{2}}{\sigma \sqrt{\pi}} \exp \left(-\frac{x^2}{2 \sigma^2}\right)
\end{equation}
\begin{equation}
%halfnormal ccdf
\erf\left(\frac{x}{\sigma \sqrt{2}}\right)
\end{equation}
\begin{equation}
%lognormal pdf
\frac{1}{x \sigma \sqrt{2 \pi}}\ e^{- \frac{(\ln\ x - \mu)^2}{2\sigma^2}}
\end{equation}
\begin{equation}
%lognormal ccdf
\frac{1}{2} + \frac{1}{2}\ \erf\left[\frac{\ln\ x - \mu}{\sqrt{2 \sigma}}\right]
\end{equation}

\begin{linenumbers}
To determine the parameters which best fit the distributions to the data we used maximum likelihood. This was conducted in Python using the Scipy package \textit{minimize} and the Sequential Least Squares Programming (SLSQP) method. The function minimises an objective function through gradient decent. Consequentially, we provided the method with the negative log-likelihood equations for each distribution in order to identify the parameters producing the maximum log-likelihood estimate (see supplementary equations ?, ?, ? and ?). Once the parameters with the highest likelihood are identified we fit the model to the complimentary cumulative distribution function (CCDF). This was chosen over the probability density function (PDF) as the probability of obtaining an exact value is 0, where as the CCDF allows us to view the probability of obtaining a value greater than or equal to a particular value (see supplementary equations ?, ?, ? and ?). 
\end{linenumbers}

\subsection{Sum of exponentials procedure}

\begin{linenumbers}
\hspace{\parindent}
The general approach to test for scale free movements has been proposed by \cite{Murphy2007} yet the presence of L\`evi processes remains controversial \citep{Pyke2015}. Following on from studies identifying composite Brownian motion, scale can be characterised by the rate parameter of the exponential distribution \citep{Petrovskii2011}. Consequently, Brownian motion is characterised by movement along a single exponential, scale free movement possesses a theoretically infinite number of rates and composite Brownian motion is described by a limited number of rates greater than 1. 

The sum of exponentials (SumExp) procedure used herein is composed of $n$ exponential functions (equation ?) summed together, each constrained by a weighting factor (equation ?, subject to equations ? and ?). Consequently, this procedure provides a continuum from Brownian to L\`evy processes. In accordance with \cite{Murphy2007}, our approach uses likelihood based methods to fit the SumExp. However, unlike other studies \citep{Petrovskii2011, Sakamoto2017, Gautestad2012, Zhao2016} the strength of our method comes in its ability to visualise data in terms of scale rather than arbitrarily fitting discrete scaling models.

The parameters of the model are $\lambda$ and $\psi$, where $\lambda$ is the rate component of the exponential and $\psi$ is the weighting factor which influences the relative contribution of an exponential with a given rate in the sum. $\lambda$ values are restricted to only positive real numbers (equation ?) and both individual $\psi$ and the combined sum of $\psi$ values must be greater than or equal to 0 and less than or equal to 1 (equation ?).
\end{linenumbers}

\begin{equation}
%single exponential equation
\lambda e^{-\lambda x}
\end{equation}
\begin{equation}
% sum of exponentials equation
f(x) = \sum_{i=1}^{n-1} \psi_i \lambda_i e^{-\lambda_i x} + \left(1 - \sum_{i=1}^{n-1}\psi_i\right) \lambda_n e^{-\lambda_n x}
\end{equation}
subject to (eq ?, ? \& ?)
\begin{equation}
% constraints on individual psi
0\leq \psi_i \leq 1
\end{equation}
\begin{equation}
% constraints on sum of psi
0\leq \sum_{i=1}^{n-1}\psi_i \leq 1
\end{equation}
\begin{equation}
% constraints on individual lambda
\lambda = \mathbb{R}^+
\end{equation}

\subsection{Numerical optimisation}

\begin{linenumbers}
\hspace{\parindent}
In order to identify the most probable parameter values for the SumExp model we used numerical optimisation to estimate the parameters with maximum likelihood. The procedure for doing this is to take the product of Sumexp over the number of observations (equation ?) and find the peak in the resulting likelihood curve. For easier computation the log-likelihood is used herein (equation ?). The analysis was conducted in Python using the scipy.optimize package \textit{minimize} (ref) and the Sequential Least Squares Programming (SLSQP) routine (ref) to identify the minima of the negative log-likelihood function. The SLSQP method uses gradient decent to scale the likelihood curve until the model derivative is within a given tolerance of 0. As such, the gradient function of partial derivatives (equation ?) is provided to improve convergence.
\end{linenumbers}
\begin{equation}
%likelihood equation for SumExp
\Lagr_{(\psi|x)} = \prod_{i=1}^{n} f(x)
\end{equation} 
\begin{equation}
%loglikelihodd equation for SumExp
\ell_{(\psi|x)} = \sum_{i=1}^{n} \ln f(x)
\end{equation} 
\begin{equation}
% gradient function of partial derivitives of SumExp. 
\nabla h(x) = \begin{bmatrix} -\sum_{i=1}^{n} [\frac{\lambda_0 e^{-\lambda_0 x_i}}{f(x_i)} - \frac{\lambda_n e^{-\lambda_n x_i}}{f(x_i)}] \\
... \\
... \\
-\sum_{i=1}^{n} [\frac{\lambda_{n-1} e^{-\lambda_{n-1} x_i}}{f(x_i)} - \frac{\lambda_n e^{-\lambda_n x_i}}{f(x_i)}] \\
\end{bmatrix}
\end{equation}

The process of numerical optimisation was the most time consuming aspect of this study. With an infinite number of possible rates and weights the parameter space for this problem is extensive in $n$ dimensions. Consequently, the success of our SumExp method lies in the development of an optimisation routine which could scale such a massive parameter space without violating constrains. To achieve this I started the routine by evaluating the likelihood landscape with just two parameters. From the information yielded from this I extended the method to $n$ exponentials and implemented various constrained and non-constrained optimisation routines  over multiple combinations of parameters. From the results returned the optimiser was tweaked until it identified the values with the best likelihood. The method tolerance remains one area where the optimiser struggled and more time is required to tune this to work for all possible model instances.

\begin{linenumbers}
The optimisation procedure used involves selecting multiple rates evenly spaced over a given interval to create a SumExp of a given size. These rates are then fixed in the model but the weights are left as free parameters within their bounds and constraints. If the rates are free in the model along with the weights then the procedure would identify the most probable sum of exponentials explaining the data. Instead, by keeping the rates fixed we allow the weights to act as switches that limit the contribution of an associated rate. By varying the upper and lower bounds as well as the number of rates (resolution) the procedure can be used to explore the data in terms of scale and aid the development of more parsimonious models. 
\end{linenumbers}

\subsection{Exploration of tail data}

\begin{linenumbers}
\hspace{\parindent}
Honey bee waggle dances are converted into distances by measuring the duration of the waggle run with a conversion rate of approximately 1 Km per second of dance (cite). Consequentially, for short distances (waggles of < 1 second) it is not known if movements interoperated as waggle dances are accurate and not in fact misidentified other movements as our data was generated from a third party. To account for this possible measurement error, data points bellow 1 Km were removed and the remaining distances corrected by subtracting one for both locations, as outlined in \cite{Petrovskii2011}. Upon this truncated data the analysis was then repeated using the same procedures outlined in the methods.
\end{linenumbers}
\section{Results}

\noindent
\subsection{Simulated data}

\subsubsection{Single exponential}

\begin{linenumbers}
\hspace{\parindent}
In order to test if the numerical method can identify the sum of exponentials with the most likely parameter estimates, we tested data sampled from a single exponential with $\lambda = 1.8$. Sampling 1,000 data points we calculated the analytical rate parameter, derived as the reciprocal of the data mean, as 1.82 with an associated maximum likelihood estimate (MLE) of -401.28. The sum of exponentials method identified a distribution with a single dominant cluster of 2 peaks at rates 1.83 and 1.85 and associated weights of 0.93 and 0.059; explaining 99\% of the observations. The remaining 1\% is explained by a peak at $\lambda = 0.86$ (figure ?). The associated MLE is -401.26. Therefore, although the likelihood is slightly improved, the added parameters indicate the single exponential is the more parsimonious model.
\end{linenumbers}
\begin{figure}[H]
	\centering
	\includegraphics[scale=1]{../Results/Plots/SyntheticData_1exp.pdf}
	\caption{Scale spectrum of linearly selected rates for data sampled from a single exponential with $\lambda = 1.8$.}
\end{figure}


\subsubsection{Multiple exponentials}

\noindent
\textbf{Equal weighting (50/50)}
\begin{linenumbers}
	
In order to test if the method can identify data derived from processes operating at two different scales, we sampled data points from a sum of two exponentials with rates 1.8 and 4.4 and equal weights. With a sample size of 1,000 the method identified five rates in 3 district clusters with $\lambda$ approximately 1.26, 2.68 and 6.49 and associated $\psi$ of approximately 0.11, 0.78 and 0.11 (Table ?, figure ?). With 10,000 observations the rates form two main clusters around rates 1.87 and 4.6 with associated $\psi$ values of 0.55 and 0.42. At 100,000 observations the rate peaks appear close to the actual input values with $\psi$ matching the input weightings (table ?).
\end{linenumbers}

\begin{table}[H]
	\centering
	\caption{Numerically optimised rates ($\lambda$) and weights ($\psi$) with data sampled from $n$ observations of a sum of two exponentials with $\lambda = 1.8,\ 4.4$ and $\psi = 0.5$.}
	\begin{tabular}{rrr}
\toprule
    $n$ &  $\lambda$ &    $\psi$ \\
\midrule
   1000 &       1.24 &  0.003290 \\
   1000 &       1.30 &  0.111582 \\
   1000 &       2.68 &  0.777452 \\
   1000 &       6.46 &  0.002152 \\
   1000 &       6.52 &  0.105525 \\
  10000 &       1.00 &  0.003992 \\
  10000 &       1.84 &  0.144445 \\
  10000 &       1.90 &  0.416288 \\
  10000 &       4.60 &  0.415916 \\
  10000 &       9.94 &  0.019359 \\
 100000 &       1.78 &  0.280559 \\
 100000 &       1.84 &  0.218817 \\
 100000 &       4.36 &  0.304631 \\
 100000 &       4.42 &  0.195990 \\
\bottomrule
\end{tabular}

\end{table}


\begin{figure}[H]
	\centering
	\includegraphics[scale=1]{../Results/Plots/SyntheticData_2exp1844.pdf}
	\caption{Scale spectrum of linearly selected rates for data sampled from a sum of two exponentials with $\lambda = 1.8,\ 4.4$ and $\psi = 0.5$. Upper bound $= 10$, lower bound $= 0$, resolution $= 150$ \textbf{A}) $n = 1,000$, \textbf{B}) $n = 10,000$,  \textbf{C}) $n = 100,000$.}
\end{figure}

\noindent
\textbf{Unequal weighting (70/30)}
\begin{linenumbers}

Using the same parameters as above but with unequal weightings of 0.3 and 0.7 for rates 1.8 and 4.4 respectively, we again tested how the method performs with different data sizes. With a data size of 1,000 the method identified two main peaks around 1.6 and 3.85 with corresponding $\psi$ values of 0.18 and 0.73. For 10,000 observations the main peaks are at 2.02 and 4.09 with associated $\psi$ of 0.33 and 0.6 with the remainder at rates 1.0 and 9.94. With 100,000 observations the peaks lie at 1.71 and 4.39, just as with equal weighting (figure ?, table?) but the weighting is 0.3 and 0.7 respectively (figure?, table?).
\end{linenumbers}
\begin{table}[H]
	\centering
	\caption{Numerically optimised rates ($\lambda$) and weights ($\psi$) with data sampled from $n$ observations of a sum of two exponentials with $\lambda = 1.8,\ 4.4$ and $\psi = 0.5$.}
	\begin{tabular}{rrr}
\toprule
    $n$ &  $\lambda$ &    $\psi$ \\
\midrule
   1000 &       1.54 &  0.047849 \\
   1000 &       1.60 &  0.129053 \\
   1000 &       3.82 &  0.352723 \\
   1000 &       3.88 &  0.382955 \\
   1000 &       3.94 &  0.087421 \\
  10000 &       1.00 &  0.008807 \\
  10000 &       2.02 &  0.331551 \\
  10000 &       4.06 &  0.238486 \\
  10000 &       4.12 &  0.360444 \\
  10000 &       9.94 &  0.060711 \\
 100000 &       1.78 &  0.248727 \\
 100000 &       1.84 &  0.048693 \\
 100000 &       4.36 &  0.543943 \\
 100000 &       4.42 &  0.155287 \\
 100000 &       9.94 &  0.003350 \\
\bottomrule
\end{tabular}

\end{table}

\begin{figure}[H]
	\centering
	\includegraphics[scale=1]{../Results/Plots/SyntheticData_2exp1844_73.pdf}
	\caption{Scale spectrum of linearly selected rates for data sampled from a sum of two exponentials with $\lambda = 1.8,\ 4.4$ and $\psi = 0.5$. Upper bound $= 10$, lower bound $= 0$, resolution $= 150$ \textbf{A}) $n = 1,000$, \textbf{B}) $n = 10,000$,  \textbf{C}) $n = 100,000$.}
\end{figure}

\noindent
\textbf{Five rate sum of exponentials}
\begin{linenumbers}

In order to test how the method responds to processes operating at multiple scales we sampled data from a five rate sum of exponential. As the previous results indicate the method can over fit to data, we used very different rates to represent processes with multiple scales scanning several orders of magnitude. The rates used are 1.8, 4.4, 7.5, 12.5 and 16.7 with equal weightings of 0.2. With a sample size of 1,000 the method identified 4 peaks with rates of 1.5, 2.74, 9.65 and 22.5 and corresponding $\psi$ of 0.05, 0.29, 0.38 and 0.28 (table ?, figure ?). 10,000 observations returned 3 main peaks at rates 2.15, 4.89 and 12.6 with corresponding $\psi$ values of 0.24, 0.23, and 0.48, with the remainder located at two other peaks of 1.0 and 29.9. With 100,000 observations the method identified four main peaks at rates 1.79, 4.3, 7.34 and 15.43 with corresponding $\psi$ values of 0.21, 0.14, 0.29 and 0.35 (table ? figure ?). 
\end{linenumbers}
\begin{table}[H]
	\centering
	\caption{Numerically optimised rates ($\lambda$) and weights ($\psi$) with data sampled from $n$ observations of a sum of five exponentials with $\lambda = 1.8,\ 4.4,\ 7.5,\ 12.5,\ 16.7$ and $\psi = 0.2$.}
	\begin{tabular}{rrr}
\toprule
    $n$ &  $\lambda$ &    $\psi$ \\
\midrule
   1000 &   1.497143 &  0.050242 \\
   1000 &   2.740000 &  0.285434 \\
   1000 &   9.617143 &  0.128375 \\
   1000 &   9.700000 &  0.195719 \\
   1000 &   9.782857 &  0.181994 \\
   1000 &   9.865714 &  0.094545 \\
   1000 &  22.377143 &  0.003371 \\
   1000 &  22.460000 &  0.020353 \\
   1000 &  22.542857 &  0.019953 \\
   1000 &  22.625714 &  0.015760 \\
   1000 &  22.708571 &  0.004230 \\
  10000 &   1.000000 &  0.009865 \\
  10000 &   2.077143 &  0.045273 \\
  10000 &   2.160000 &  0.194265 \\
  10000 &   4.894286 &  0.234190 \\
  10000 &  12.517143 &  0.080454 \\
  10000 &  12.600000 &  0.216137 \\
  10000 &  12.682857 &  0.188127 \\
  10000 &  29.917143 &  0.031690 \\
 100000 &   1.745714 &  0.006408 \\
 100000 &   1.828571 &  0.203963 \\
 100000 &   4.314286 &  0.126473 \\
 100000 &   4.397143 &  0.017868 \\
 100000 &   7.214286 &  0.001503 \\
 100000 &   7.297143 &  0.146539 \\
 100000 &   7.380000 &  0.143362 \\
 100000 &  15.334286 &  0.050094 \\
 100000 &  15.417143 &  0.136866 \\
 100000 &  15.500000 &  0.132326 \\
 100000 &  15.582857 &  0.034596 \\
\bottomrule
\end{tabular}

\end{table}
\begin{figure}[H]
	\centering
	\includegraphics[scale=1]{../Results/Plots/SyntheticData_2exp1844_5.pdf}
	\caption{Scale spectrum of linearly selected rates for data sampled from a sum of five exponentials with $\lambda = 1.8,\ 4.4,\ 7.5,\ 12.5,\ 16.7$ and $\psi = 0.2$. Upper bound $= 30$, lower bound $= 1$, resolution $= 350$. \textbf{A}) $n = 1,000$, \textbf{B}) $n = 10,000$,  \textbf{C}) $n = 100,000$.}
\end{figure}
\begin{linenumbers}

What is notable here is that the method was unable to identify the input rates and appropriate weights at any of the data sizes tested. Although they converged towards to the true values as sample size was increased, the procedure omitted the rate 12.5. This indicates the method struggles to identify process occurring at more than 3 or 4 scales without very large data sets, possibly due to overlaps in parameter space (figure ?).
\end{linenumbers}
\begin{figure}[H]
	\centering
	\includegraphics[scale=1]{../Results/Plots/SyntheticData_Hist_5.pdf}
	\caption{Overlapping histograms of data sampled from exponentials with different rates. \textbf{A}) $\lambda = 1.8, 4.4$, \textbf{B}) $\lambda = 1.8, 4.4, 7.5$, \textbf{C}) $\lambda = 1.8, 4.4, 7.5, 12.5$, \textbf{D}) $\lambda = 1.8, 4.4, 7.5, 12.5, 16.7$.}
\end{figure}


\subsection{Analysis of foraging distance}

\begin{linenumbers}
\hspace{\parindent}
The agri-rural data consists of 193 observations ranging from 0.016 to 5.17Km with a mean of 1.29. The urban data consists of 221 observations ranging from 0.0006 to 3.18Km with a mean of 0.85. Combined these datasets consist of 414 observations ranging from 0.0006 to 5.17Km with a mean of 1.05.
\end{linenumbers}

\begin{figure}[H]
	\centering
	\includegraphics[scale=1]{../Results/Plots/DistributionHist.pdf}
	\caption{Histogram of agri-rural (\textbf{A}), urban (\textbf{B}) and combined (\textbf{C}) foraging distances.}
\end{figure}
\begin{linenumbers}

With values bellow 1Km removed and the remaining data normalised by subtracting 1 from the distances, the agri-rural data consists of 107 observations ranging from 0.025 to 4.17Km with a mean of 0.85. The urban data consists of 81 observations ranging from 0.006 to 2.18Km with a mean of 0.53. Combined these datasets consist of 188 observations ranging from 0.0006 to 4.17Km with a mean of 0.71.
\end{linenumbers}
\begin{figure}[H]
	\centering
	\includegraphics[scale=1]{../Results/Plots/DistributionHist_1km.pdf}
	\caption{Histogram of agri-rural (\textbf{A}), urban (\textbf{B}) and combined (\textbf{C}) foraging distances above 1Km, normalised by subtracting 1Km from results.}
\end{figure}
\begin{linenumbers}

In order to accommodate for the difference in distributions between these two data sets, foraging distances were compared using a bootstrapped hypothesis test. This was chosen as a non-parametric equivalent to the t-test as the data violates the condition of normality (figure ? histogram), whilst retaining a high degree of statistical power (ref on bootstrapping). For both the truncated and full data sets, foraging distance differed significantly between environments (Non-parametric bootstrap test: simulations = 10,000. full data: agri-rural mean = 1.29, urban mean 0.85, degrees of freedom = 192 and 220, t = 5.70, p $\textless$ 0.001; truncated data: agri-rural mean = 0.85, urban mean 0.53, degrees of freedom = 106 and 80, t = 3.34, p $\textless$ 0.01) (figure ?). 
\end{linenumbers}
\begin{figure}[H]
	\centering
	\includegraphics[scale=0.9]{../Results/Plots/DistDiff.pdf}
	\caption{Comparison of agri-rural and urban mean foraging distances. With both datasets foraging distance is significantly different between environments (Non-parametric bootstrap test: simulations = 10,000) \textbf{A}) Full data, agri-rural mean = 1.29, urban mean 0.85, degrees of freedom = 192 and 220, t = 5.70, p $\textless$ 0.001. \textbf{B}) Truncated data, agri-rural mean = 0.85, urban mean 0.53, degrees of freedom = 106 and 80, t = 3.34, p $\textless$ 0.01.}
\end{figure}

\subsection{Fitting distributions using maximum likelihood}

\subsubsection{Full data}

\begin{linenumbers}
\hspace{\parindent}
In order to determine the distribution which best describes the data, maximum likelihood methods were used to fit a number of candidate distributions. Each distribution was fit to the urban and agri-rural data and the fit assessed using the Akaike information criterion (AIC) and associated Akaike weights (AICw). For the rural data the most parsimonious model is the gamma distribution (table ?) with an AIC score seven points lower than the next best fitting model, the half-normal distribution. Using Akaike weights we can determine the probability that a given model is the best model of our candidate set. By dividing the AICw of the best model by that of the next best candidate (0.978/0.0233), the Gamma distribution is approximately 41.9 times more likely to be the best model in terms of the Kullback–Leibler discrepancy than the half-normal model.

For the urban data, the best fitting model is the half-normal distribution with an AIC score 14 points lower than the next best fitting model, the gamma distribution (table ?, figure ?). The interpretation of Akaike weights indicates the half-normal is approximately 1225.5 times more likely to represent the best model in terms of the Kullback–Leibler discrepancy than the Gamma distribution.

For the agri-rural and urban foraging data combined, the best fitting model is the half-normal with an AIC score 10 points lower than the next best fitting model, the gamma distribution (table ?, figure ?). The interpretation of Akaike weights indicates the half-normal is approximately 196.1 times more likely to represent the best model in terms of the Kullback–Leibler discrepancy than the Gamma distribution.
\end{linenumbers}

\begin{table}[H]
	\centering
	\caption{AIC and weighted AIC scores for distributions fit using maximum likelihood to Agri-rural foraging distances.}
	\begin{tabular}{llrr}
\toprule
Distribution &      MLE &         AIC &          AICw \\
\midrule
       Gamma & -222.258 &  448.516283 &  9.766669e-01 \\
 Half-normal & -226.992 &  455.984833 &  2.333305e-02 \\
 Exponential & -242.106 &  486.211336 &  6.373377e-09 \\
   Lognormal & -244.337 &  492.674635 &  2.516994e-10 \\
\bottomrule
\end{tabular}

\end{table}
\begin{table}[H]
	\centering
	\caption{AIC and weighted AIC scores for distributions fit using maximum likelihood to urban foraging distances.}
	\begin{tabular}{llrr}
\toprule
Distribution &      MLE &         AIC &          AICw \\
\midrule
 Half-normal & -174.376 &  350.751452 &  9.991404e-01 \\
       Gamma & -180.487 &  364.973596 &  8.153183e-04 \\
 Exponential & -184.401 &  370.801177 &  4.424703e-05 \\
   Lognormal & -215.753 &  435.506022 &  3.939176e-19 \\
\bottomrule
\end{tabular}

\end{table}
\begin{table}[H]
	\centering
	\caption{AIC and weighted AIC scores for distributions fit using maximum likelihood to combined argi-rural and urban foraging distances.}
	\begin{tabular}{llrr}
\toprule
Distribution &      MLE &         AIC &          AICw \\
\midrule
 Half-normal & -416.694 &  835.388612 &  9.949263e-01 \\
       Gamma & -420.973 &  845.945810 &  5.073695e-03 \\
 Exponential & -435.614 &  873.228880 &  6.037832e-09 \\
   Lognormal &  -483.13 &  970.260184 &  5.138081e-30 \\
\bottomrule
\end{tabular}

\end{table}

\begin{figure}[H]
	\centering
	\includegraphics[scale=1]{../Results/Plots/RuralDistributionFit.pdf}
	\caption{Complementary Cumulative Distribution Frequency (CCDF) of agri-rural foraging distances with fitted model lines. \textbf{A}) Exponential, \textbf{B}) Gamma, \textbf{C}) Half-normal, \textbf{D}) Lognormal.}
\end{figure}

\begin{figure}[H]
	\centering
	\includegraphics[scale=1]{../Results/Plots/UrbanDistributionFit.pdf}
	\caption{Complementary Cumulative Distribution Frequency (CCDF) of urban foraging distances with fitted model lines. \textbf{A}) Exponential, \textbf{B}) Gamma, \textbf{C}) Half-normal, \textbf{D}) Lognormal.}
\end{figure}

\begin{figure}[H]
	\centering
	\includegraphics[scale=1]{../Results/Plots/AllDistributionFit.pdf}
	\caption{Complementary Cumulative Distribution Frequency (CCDF) of combined agri-rural and urban foraging distances with fitted model lines. \textbf{A}) Exponential, \textbf{B}) Gamma, \textbf{C}) Half-normal, \textbf{D}) Lognormal.}
\end{figure}

\subsubsection{Truncated data}

\begin{linenumbers}
\hspace{\parindent}
With distances lower than one Kilometre removed and the remaining distances normalised by subtracting 1, the agri-rural data shows the best fitting model is the gamma. However the difference in AIC is less than 1 point to the next best fitting model, the exponential (table ?). A comparison of the weighted AIC scores suggests the gamma is approximately 1.4 times more likely to be most parsimonious model than the exponential (table ?).

In contrast, the urban data shows the best model as the exponential, almost 2 AIC points lower than the next best fitting model, the gamma distribution (table ?). The comparison of AIC weights indicates the exponential is approximately 2.5 times more likely to be the best model than the gamma (table ?).

For both the agri-rural and urban foraging distances combined, the best model was identified as the exponential, 2 AIC points lower than the next best fitting model, the gamma distribution (table ?). The comparison of AIC weights indicates the exponential is approximately 2.5 times more likely to be the best model than the gamma (table ?). 
\end{linenumbers}

\begin{table}[H]
	\centering
	\caption{AIC and weighted AIC scores for distributions fit using maximum likelihood to Agri-rural foraging data greater than 1Km.}
	\begin{tabular}{llrr}
\toprule
Distribution &      MLE &         AIC &      AICw \\
\midrule
       Gamma & -88.4953 &  180.990608 &  0.444697 \\
 Exponential & -89.8349 &  181.669702 &  0.316665 \\
   Lognormal & -89.1227 &  182.245326 &  0.237468 \\
 Half-normal & -95.4361 &  192.872228 &  0.001170 \\
\bottomrule
\end{tabular}

\end{table}
\begin{table}[H]
	\centering
	\caption{AIC and weighted AIC scores for distributions fit using maximum likelihood to urban foraging data greater than 1Km.}
	\begin{tabular}{llrr}
\toprule
Distribution &      MLE &        AIC &      AICw \\
\midrule
 Exponential &  -29.446 &  60.891934 &  0.699804 \\
       Gamma & -29.3618 &  62.723513 &  0.280062 \\
 Half-normal & -32.9953 &  67.990502 &  0.020116 \\
   Lognormal &  -39.024 &  82.048053 &  0.000018 \\
\bottomrule
\end{tabular}

\end{table}
\begin{table}[H]
	\centering
	\caption{AIC and weighted AIC scores for distributions fit using maximum likelihood to combined argi-rural and urban foraging distances greater than 1Km.}
	\begin{tabular}{llrr}
\toprule
Distribution &      MLE &         AIC &          AICw \\
\midrule
 Exponential & -124.347 &  250.693054 &  7.134821e-01 \\
       Gamma & -124.259 &  252.517781 &  2.865158e-01 \\
   Lognormal & -136.396 &  276.792210 &  1.534700e-06 \\
 Half-normal & -138.436 &  278.871001 &  5.427747e-07 \\
\bottomrule
\end{tabular}

\end{table}

\begin{figure}[H]
	\centering
	\includegraphics[scale=1]{../Results/Plots/RuralDistributionFit_1km.pdf}
	\caption{Complementary Cumulative Distribution Frequency (CCDF) of agri-rural foraging distances over 1Km with fitted model lines. \textbf{A}) Exponential, \textbf{B}) Gamma, \textbf{C}) Half-normal, \textbf{D}) Lognormal.}
\end{figure}
\begin{figure}[H]
	\centering
	\includegraphics[scale=1]{../Results/Plots/UrbanDistributionFit_1km.pdf}
	\caption{Complementary Cumulative Distribution Frequency (CCDF) of urban foraging distances over 1Km with fitted model lines. \textbf{A}) Exponential, \textbf{B}) Gamma, \textbf{C}) Half-normal, \textbf{D}) Lognormal.}
\end{figure}
\begin{figure}[H]
	\centering
	\includegraphics[scale=1]{../Results/Plots/AllDistributionFit_1km.pdf}
	\caption{Complementary Cumulative Distribution Frequency (CCDF) of combined agri-rural and urban foraging distances over 1Km with fitted model lines. \textbf{A}) Exponential, \textbf{B}) Gamma, \textbf{C}) Half-normal, \textbf{D}) Lognormal.}
\end{figure}

\subsection{Sum of exponential fitting}

\subsubsection{Full data}
\begin{linenumbers}
\hspace{\parindent}
For the agri-rural data, using variations of fixed rates between 0 and 100, the method identified a single exponential with $\lambda = 0.77$ most likely contributed to the data. This matches the analytically derived single exponential rate value (figure ?, table ?). For the urban data a single peak is also identified around 1.18 again matching the analytical MLE for a single exponential (figure ?, table ?). When the data from each site is combined, the method identified a single peak at $\lambda = 0.94$ explained the data best, within 0.01 of the analytical MLE of the combined data (table ?, figure ?). 
\end{linenumbers}

\begin{table}[H]
	\centering
	\caption{Estimated rate ($\lambda$) and weight ($\psi$) sum of exponential parameters for agri-rural and urban foraging distances. Analytical $\lambda$ derived from MLE of single exponential.}
	\begin{tabular}{lrrr}
\toprule
   Location &  $\lambda$ &    $\psi$ &  Analytical $\lambda$ \\
\midrule
 Agri-rural &   0.771429 &  0.916771 &              0.775356 \\
 Agri-rural &   0.777143 &  0.083229 &              0.775356 \\
      Urban &   1.171429 &  0.084733 &              1.180111 \\
      Urban &   1.177143 &  0.915253 &              1.180111 \\
   Combined &   0.948571 &  1.000000 &              0.949131 \\
\bottomrule
\end{tabular}

\end{table}


\begin{figure}[H]
	\centering
	\includegraphics[scale=1]{../Results/Plots/Sumexp_rural_full.pdf}
	\caption{\textbf{A}) Scale spectrum of agri-rural foraging distances. Single peak identified with $\lambda = 0.77$, upper bound $= 2$, lower bound $= 0$, resolution $= 350$. \textbf{B}) Complementary Cumulative Distribution Frequency (CCDF) of agri-rural foraging distances. Black line is data, blue line is sum of exponentials fit.}
\end{figure}

\begin{figure}[H]
	\centering
	\includegraphics[scale=1]{../Results/Plots/Sumexp_urban_full.pdf}
	\caption{\textbf{A}) Scale spectrum of urban foraging distances. Single peak identified with $\lambda = 1.18$, upper bound $= 2$, lower bound $= 0$, resolution $= 350$. \textbf{B}) Complementary Cumulative Distribution Frequency (CCDF) of urban foraging distances. Black line is data, blue line is sum of exponentials fit.}
\end{figure}

\begin{figure}[H]
	\centering
	\includegraphics[scale=1]{../Results/Plots/Sumexp_comb_full.pdf}
	\caption{\textbf{A}) Scale spectrum of combined agri-rural and urban foraging distances. Single peak identified with $\lambda = 0.94$, upper bound $= 4$, lower bound $= 0$, resolution $= 350$. \textbf{B}) Complementary Cumulative Distribution Frequency (CCDF) of urban foraging distances. Black line is data, blue line is sum of exponentials fit.}
\end{figure}


\subsubsection{Truncated data}

\begin{linenumbers}
\hspace{\parindent}
For the agri-rural foraging data over 1Km, using variations of fixed rates between 0 and 100, the method identified a single peak around $\lambda = 1.17$ most likely contributed to the data. This matches the analytically derived single exponential rate value (figure ?, table ?). For the urban foraging data over 1Km the method identified two peaks around rates 1.86 and 46.9. The peak around 1.86 is the dominant peak, explaining 97\% of the data with the remainder produced by the secondary peak. Consequentially, the second peak is small in figure ? (figure ?, table ?). The dominant peak is 0.03 larger than the analytical MLE of a single exponential for this data, 1.189. The associated likelihoods between the sum of exponentials and the single exponential is -29.081 and -29.446 respectively, however, this is not significantly different and indicates the single exponential is the more parsimonious model (Sum of exponentials $MLE = -29.08\ AIC = 76.2, AIC_w < 0.001$, single exponential $MLE = -29.46\ AIC = 60.9, AIC_w > 0.999$, table ?). 

Combined, the foraging data from both sites showed two peaks around rates 1.36 and 2.08, explaining 90\% and 10\% of the data respectively (table ?, figure ?). The likelihood for the sum of exponential and a single exponential is 0.02 higher than for the single exponential, however this is not significant indicating the single exponential is the more parsimonious model (SumExp $MLE = -124.33,\ AIC = 284.67,\ AIC_w < 0.001$, single exponential $MLE = -124.35,\ AIC = 250.69,\ AIC_w > 0.999$, table ?).
\end{linenumbers}

\begin{table}[H]
	\centering
	\caption{Estimated rate ($\lambda$) and weight ($\psi$) sum of exponential parameters for agri-rural, urban and combined foraging distances. Analytical $\lambda$ derived from MLE of single exponential.}
	\begin{tabular}{lrrr}
\toprule
   Location &  $\lambda$ &    $\psi$ &  Analytical $\lambda$ \\
\midrule
 Agri-rural &   1.154286 &  0.038405 &              1.174006 \\
 Agri-rural &   1.160000 &  0.158546 &              1.174006 \\
 Agri-rural &   1.165714 &  0.225814 &              1.174006 \\
 Agri-rural &   1.171429 &  0.241893 &              1.174006 \\
 Agri-rural &   1.177143 &  0.208409 &              1.174006 \\
 Agri-rural &   1.182857 &  0.126933 &              1.174006 \\
      Urban &   1.818182 &  0.913226 &              1.889797 \\
      Urban &   1.909091 &  0.055472 &              1.889797 \\
      Urban &  46.636364 &  0.002272 &              1.889797 \\
      Urban &  46.727273 &  0.003751 &              1.889797 \\
      Urban &  46.818182 &  0.004365 &              1.889797 \\
      Urban &  46.909091 &  0.004139 &              1.889797 \\
      Urban &  47.000000 &  0.004755 &              1.889797 \\
      Urban &  47.090909 &  0.007131 &              1.889797 \\
      Urban &  47.181818 &  0.004889 &              1.889797 \\
   Combined &   1.285714 &  0.008211 &              1.402957 \\
   Combined &   1.300000 &  0.067885 &              1.402957 \\
   Combined &   1.314286 &  0.106875 &              1.402957 \\
   Combined &   1.328571 &  0.128086 &              1.402957 \\
   Combined &   1.342857 &  0.134152 &              1.402957 \\
   Combined &   1.357143 &  0.127452 &              1.402957 \\
   Combined &   1.371429 &  0.110134 &              1.402957 \\
   Combined &   1.385714 &  0.087323 &              1.402957 \\
   Combined &   1.400000 &  0.063454 &              1.402957 \\
   Combined &   1.414286 &  0.047699 &              1.402957 \\
   Combined &   1.428571 &  0.024425 &              1.402957 \\
   Combined &   2.042857 &  0.006489 &              1.402957 \\
   Combined &   2.057143 &  0.014659 &              1.402957 \\
   Combined &   2.071429 &  0.019576 &              1.402957 \\
   Combined &   2.085714 &  0.020986 &              1.402957 \\
   Combined &   2.100000 &  0.018636 &              1.402957 \\
   Combined &   2.114286 &  0.012281 &              1.402957 \\
   Combined &   2.128571 &  0.001677 &              1.402957 \\
\bottomrule
\end{tabular}

\end{table}

\begin{table}[H]
	\centering
	\caption{Model statistics for urban foraging distances greater than 1Km.}
	\begin{tabular}{lrrr}
\toprule
  Model &        MLE &        AIC &      AICw \\
\midrule
    Exp & -29.445967 &  60.891934 &  0.999517 \\
 SumExp & -29.080624 &  76.161249 &  0.000483 \\
\bottomrule
\end{tabular}

\end{table}

\begin{table}[H]
	\centering
	\caption{Model statistics for combined agri-rural and urban foraging distances greater than 1Km.}
	\begin{tabular}{lrrr}
\toprule
  Model &         MLE &         AIC &          AICw \\
\midrule
    Exp & -124.346527 &  250.693054 &  1.000000e+00 \\
 SumExp & -124.334338 &  284.668675 &  4.190710e-08 \\
\bottomrule
\end{tabular}

\end{table}

\begin{figure}[H]
	\centering
	\includegraphics[scale=1]{../Results/Plots/Sumexp_rural_trun.pdf}
	\caption{\textbf{A}) Scale spectrum of agri-rural foraging distances. Single peak identified with $\lambda = 1.17$, upper bound $= 2$, lower bound $= 0$, resolution $= 350$. \textbf{B}) Complementary Cumulative Distribution Frequency (CCDF) of agri-rural foraging distances. Black line is data, blue line is sum of exponentials fit.}
\end{figure}

\begin{figure}[H]
	\centering
	\includegraphics[scale=1]{../Results/Plots/Sumexp_urban_trun.pdf}
	\caption{\textbf{A}) Scale spectrum of urban foraging distances. Two peaks identified around $\lambda = 1.86$ and $46.9$ with associated $\psi = 0.97$ and $0.03$, upper bound $= 50$, lower bound $= 0$, resolution $= 550$. \textbf{B}) Complementary Cumulative Distribution Frequency (CCDF) of urban foraging distances. Black line is data, blue line is sum of exponentials fit.}
\end{figure}

\begin{figure}[H]
	\centering
	\includegraphics[scale=1]{../Results/Plots/Sumexp_comb_trun.pdf}
	\caption{\textbf{A}) Scale spectrum of combined foraging distances from agri-rural and urban sites. Two peaks identified with $\lambda = 1.36$ and $2.08$ with associated $\psi = 0.9$ and $0.1$, upper bound $=5$, lower bound $= 0$, resolution $= 350$. \textbf{B}) Complementary Cumulative Distribution Frequency (CCDF) of combined foraging distances from agri-rural and urban sites. Black line is data, blue line is sum of exponentials fit.}
\end{figure}

\section{Discussion}

\begin{linenumbers}
\hspace{\parindent}
The aim of this study was to evaluate if there is a fundamental difference in how honey bees forage in their environment. To answer this question we provided two approaches: a classical approach of fitting distributions using Maximum likelihood and evaluating model fit using AIC, and the SumExp procedure. Whilst the classical approach identified significant differences in the foraging distances between these two environments it is unable to identify underlying processes in terms of scale. In contrast, the SumExp procedure is able to identify underlying scales in the data to infer processes from. Our results suggest honey bees in urban and agri-rural environments apply the same mechanisms of foraging, despite environmental differences. That colonies over 30,000 individuals strong show such constraint in foraging distances indicates foraging is tightly coordinated.

In fitting our selection of heavy and light tailed distributions to honey bee foraging data, we identified significant differences in the most parsimonious models in each environment. This difference is likely influenced by measurement error causing an over representation of small distance observations. With these values removed, the best fitting models are the gamma and exponential for the agri-rural and urban locations respectively. However, the difference in AIC between the gamma and exponential in the agri-rural location is statistically insignificant, indicating the exponential is the more parsimonious model, although the means are significantly different between these two data sets.

As the exponential indicates a single scale underlies this data, it cannot be determined from these fits alone if this is indeed the case. For example, when combining the data from these two environments and refitting the data with our candidate distributions, an exponential with a different mean is identified as the most parsimonious model. Despite this, we have already determined that the individual locations in isolation stem from exponential distributions which differ significantly in their means. This is not surprising as the exponential can only take a single rate parameter. This is also observed in the full data with no values removed where the classical methodology identifies the underlying distribution is the half-normal, despite our data in isolation stemming from a gamma and a half normal distribution. This captures precisely the problem we aim to solve using the SumExp method. 

By using a weighted sum of exponentials, the SumExp method is able to show the underlying patterns of the data in terms of scale. Throughout our tests with synthetic data the method was able to identify the number of exponentials used to produce the data with a large enough sample size. That the methods ability to identify the exact rates improves with sample size indicates sample size strongly influences convergence. This is likely due to the ability of the method to identify the true peak of the likelihood curve. The optimisation routine applied herein returns parameters that correspond to a peak in the likelihood curve. This peak is determined by altering parameters until the likelihood's derivative is within a certain tolerance of 0. With a small sample size the region at the peak may be relatively flat. Increasing sample size therefore magnifies this peak allowing the tolerance of the method to rule out similar distributions and find the true peak easier. More research is required to decrease this tolerance so it can converge with lower sample sizes. Nevertheless, that the method identified the correct number of rates in all but the 5 rate sum of exponentials indicates the method is working to a high enough resolution to identify multiple underlying scales. Consequently, when applied to foraging distances we can be confident of the accuracy in the number of scales returned, if not their exact values. 

On applying the SumExp procedure to the full data set, a key limitation of the method was identified. The accumulation of short foraging distances creates a shoulder not dissimilar to a Gaussian hump. Consequentially, by using a method of exponential sums, it is not possible to recreate this attribute. Fitting the method to foraging data returns a single scale in each location and also in the combined locations, despite the results coming from alternate distributions. However, as it was identified this hump is likely the product of measurement error, where the signal to noise ratio is particularly bad, it is important to assess the models fit with such data removed.

The SumExp method applied to foraging distances over 1 Km identifies a single scale in both locations. With the combined data of each location the method is able to identify two scales. Whilst the values returned are different to the peaks identified at each location, the findings from our synthetic data tests indicates this is likely a factor of the reduced tolerance with smaller sample sizes. Nevertheless, that two scales were identified demonstrates the method can identify underlying scales in the data, reinforcing our observations of a single underlying process acting in each location. 

As this study is phenomenological, inferences about possible mechanisms are purely speculative and to pin down exact causes of these statistical patterns requires further empirical research. Nonetheless, our findings are well placed to inform future studies. The finding of a single scale in each location, despite significant differences in foraging distances between the two environments, suggests a common internal colony foraging strategy unrelated to differing environmental constraints. Due to different features of agri-rural and urban landscapes it can be expected that foraging distances should vary in response. Indeed, agri-rural honey bees forage at significantly greater distances than their urban counterparts. There are two alternative hypothesis we can draw from this observation: either resources within urban landscapes are more abundant than in agri-rural habitats, permitting the colony to forage at shorter distances, or physical features of the urban environment restrict foraging to a given area, whereas agri-rural features permit a larger foraging range.

The urban matrix is often characterised by large expanses of abiotic structures with few resources for pollinators. Despite this, urban green spaces such as parks, allotments and gardens have been shown to provide high floral resources throughout the season \citep{Baldock2015, Baldock2019, Plascencia2017} and support diverse pollinator communities \citep{Hall2017}. In addition, urban bee colonies have been found to be both more productive \citep{Lecocq2015} and have higher reproductive success \citep{Samuelson2018} than agri-rural populations. In contrast, agri-rural areas, despite appearing more pollinator friendly, impose challenges on honey bees in the form of pesticides \citep{Wood2017} and a lack of nutrient rich land \citep{Carvell2006}. Combined, the contrasting floral profiles of these landscapes may influence the disparity in foraging distances.

Alternatively, the dominance of man made surfaces may serve to constrain pollinator movement to a given area. In a study along an urban-rural gradient, \cite{Bates2011} found both pollinator abundance and diversity correlated negatively with higher levels of urbanisation and reduced floral availability. However, a similar study by \cite{Kearns2009} found such measures of urbanisation did not correlate with pollinator diversity. That these studies were conducted on pollinators in general creates problems with transposing their results on to honey bees as both papers reported some species showed the opposite trend. Consequently, to truly examine the effects in honey bees further studies are need to investigate their movement patterns and abundance as distance increases from a green space. 

Whilst it is productive to elaborate on significant differences in honey bee foraging distances between urban and agri-rural environments, our analysis of scale suggests this is more of a reactive response than a driver of underlying behaviour. The main finding from this study is the identification of movement along a single scale despite these differences, hinting at a more general underlying mechanism governing honey bee movement.

A possible mechanism underpinning such a coordinated foraging effort amongst the colony could be the genetic similarity between workers. Due to haplodiploidy, female workers are related to each other by a coefficient of 0.75 \citep{Ratnieks1989}. However, queens also mate with several males in order to increase colony genetic diversity \citep{Jones2004}. Genetic variations have been shown to influence other behaviours, such as response thresholds for temperature regulation \citep{Jones2004} and foraging choice between pollen and nectar \citep{Fewell1993, Fewell2000}, yet to our knowledge it is not known if genetic variation influences foraging distance directly. Nevertheless, our results hint that genetic diversity does not significantly vary foraging behaviour, indicating either genetic controls of foraging distances are highly conserved in honey bees or that non-genetic factors evoke greater influence.

In contrast to genetic determination, a common foraging mechanism could also be due to social cues. As a method of recruitment, the waggle dance itself could serve to constrain variations in foraging behaviour to a single scale. As recruitment is a multiplicative process, colony movement should converge to profitable patches in order to reduce the searching time of other workers \citep{Seeley1995}. Consequently, this knowledge of resource location may serve to standardise colony foraging. However, some studies have questioned the usefulness of the waggle dance as a recruitment strategy \citep{Sherman2002, Dornhaus2004, Gruter2008, Gruter2009, Schurch2013}. In a study by \cite{Sherman2002}, honey bees were denied accurate waggle dance information at different times of the year. Their results showed the waggle dance is less effective during spring and summer, when resources are more abundant and when our observations were recorded, than at other times of the year. Similarly, errors in the waggle dance and its transcription by recruits have also been shown to vary with individuals, reducing accuracy by as much as 50\% \citep{Schurch2013}. This inefficiency of the waggle dance therefore casts doubt on the overall effectiveness in guiding recruits to productive locations. With such apparent inefficiency in the waggle dance more research is required to pin point how this complex behaviour influences foraging distances.

Overall, our results of a single dominant scale suggests there exists no fundamental difference in how honey bees explore contrasting landscapes. Whilst the exact mechanisms underpinning this common strategy are unclear, future studies evaluating possible genetic or social mechanisms would benefit from including findings provided by the methods outlined herein. It would also be productive to evaluate the extent to which our findings are consistent in complex landscapes. In a mixed urban/agri-rural environment, would we observe a single scale, as in the isolated environments, or a composition of different scales? More data from urban-agri-rural gradients is needed to answer such questions, but the SumExp model developed herein, whilst requiring further calibration to tease out exact parameter values, provides the ability to dissect relatively simple foraging data in terms of scale and extract signatures of the underlying processes. The methods and optimisation routine presented here therefore moves us closer towards a general method of identifying patterns in animal movement data. As such, this method has huge potential for increasing our knowledge of foraging behaviours in central place pollinators as well as wider applications for the study of animal movement.
\end{linenumbers}

\section{Acknowledgements}

In addition to thanking my supervisors Richard Gill and Vincent Jansen for general help with this project, I also thank Lucas Dias Fernandes for his elegant explanations of specific mathematical concepts. Andres Arce and Peter Graystock also deserve a mention for tolerating my incessant bee related questions and for providing much needed coffee.

\newpage
\section{Data and Code Availability}
For the purposes of reproducible science, all data and code for this project is available at \textit{https://github.com/joseph-palmer/CMEECourseWork/tree/master/Project}. Please see the relevant README file given on the project GitHub page for details of how to run certain scripts.

As this data remains the property of Dr. Elli Leadbeater of Royal Holloway University of London, its use for anything other than recreating the results of this study is prohibited. For details on this data, please contact Elli Leadbeater \textit{(Elli.Leadbeater@rhul.ac.uk)}.

\newpage

\bibliographystyle{plainnat}
\bibliography{Project_bib}
\end{document}





